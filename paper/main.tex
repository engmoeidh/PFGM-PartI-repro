%Version 3.1 December 2024
% See section 11 of the User Manual for version history
%
%%%%%%%%%%%%%%%%%%%%%%%%%%%%%%%%%%%%%%%%%%%%%%%%%%%%%%%%%%%%%%%%%%%%%%
%%                                                                 %%
%% Please do not use \input{...} to include other tex files.       %%
%% Submit your LaTeX manuscript as one .tex document.              %%
%%                                                                 %%
%% All additional figures and files should be attached             %%
%% separately and not embedded in the \TeX\ document itself.       %%
%%                                                                 %%
%%%%%%%%%%%%%%%%%%%%%%%%%%%%%%%%%%%%%%%%%%%%%%%%%%%%%%%%%%%%%%%%%%%%%

%%\documentclass[referee,sn-basic]{sn-jnl}% referee option is meant for double line spacing

%%=======================================================%%
%% to print line numbers in the margin use lineno option %%
%%=======================================================%%

%%\documentclass[lineno,pdflatex,sn-basic]{sn-jnl}% Basic Springer Nature Reference Style/Chemistry Reference Style

%%=========================================================================================%%
%% the documentclass is set to pdflatex as default. You can delete it if not appropriate.  %%
%%=========================================================================================%%

%%\documentclass[sn-basic]{sn-jnl}% Basic Springer Nature Reference Style/Chemistry Reference Style

%%Note: the following reference styles support Namedate and Numbered referencing. By default the style follows the most common style. To switch between the options you can add or remove “Numbered” in the optional parenthesis.
%%The option is available for: sn-basic.bst, sn-chicago.bst%

%%\documentclass[pdflatex,sn-nature]{sn-jnl}% Style for submissions to Nature Portfolio journals
%%\documentclass[pdflatex,sn-basic]{sn-jnl}% Basic Springer Nature Reference Style/Chemistry Reference Style
\documentclass[pdflatex,sn-mathphys-num]{sn-jnl}% Math and Physical Sciences Numbered Reference Style
%%\documentclass[pdflatex,sn-mathphys-ay]{sn-jnl}% Math and Physical Sciences Author Year Reference Style
%%\documentclass[pdflatex,sn-aps]{sn-jnl}% American Physical Society (APS) Reference Style
%%\documentclass[pdflatex,sn-vancouver-num]{sn-jnl}% Vancouver Numbered Reference Style
%%\documentclass[pdflatex,sn-vancouver-ay]{sn-jnl}% Vancouver Author Year Reference Style
%%\documentclass[pdflatex,sn-apa]{sn-jnl}% APA Reference Style
%%\documentclass[pdflatex,sn-chicago]{sn-jnl}% Chicago-based Humanities Reference Style

%%%% Standard Packages
%%<additional latex packages if required can be included here>

\usepackage{graphicx}%
\usepackage{multirow}%
\usepackage{amsmath,amssymb,amsfonts}%
\usepackage{amsthm}%
\usepackage{mathrsfs}%
\usepackage[title]{appendix}%
\usepackage{xcolor}%
\usepackage{textcomp}%
\usepackage{manyfoot}%
\usepackage{booktabs}%
\usepackage{algorithm}%
\usepackage{algorithmicx}%
\usepackage{algpseudocode}%
\usepackage{listings}%
%%%%
\graphicspath{{./Fig/}}
\usepackage{longtable,booktabs}
\newcommand{\orcidd}[1]{\href{https://orcid.org/#1}{\includegraphics[width=8pt]{orcid}}}


%%%%%=============================================================================%%%%
%%%%  Remarks: This template is provided to aid authors with the preparation
%%%%  of original research articles intended for submission to journals published
%%%%  by Springer Nature. The guidance has been prepared in partnership with
%%%%  production teams to conform to Springer Nature technical requirements.
%%%%  Editorial and presentation requirements differ among journal portfolios and
%%%%  research disciplines. You may find sections in this template are irrelevant
%%%%  to your work and are empowered to omit any such section if allowed by the
%%%%  journal you intend to submit to. The submission guidelines and policies
%%%%  of the journal take precedence. A detailed User Manual is available in the
%%%%  template package for technical guidance.
%%%%%=============================================================================%%%%

%% as per the requirement new theorem styles can be included as shown below
\theoremstyle{thmstyleone}%
\newtheorem{theorem}{Theorem}%  meant for continuous numbers
%%\newtheorem{theorem}{Theorem}[section]% meant for sectionwise numbers
%% optional argument [theorem] produces theorem numbering sequence instead of independent numbers for Proposition
\newtheorem{proposition}[theorem]{Proposition}%
%%\newtheorem{proposition}{Proposition}% to get separate numbers for theorem and proposition etc.

\theoremstyle{thmstyletwo}%
\newtheorem{example}{Example}%
\newtheorem{remark}{Remark}%

\theoremstyle{thmstylethree}%
\newtheorem{definition}{Definition}%

\raggedbottom
%%\unnumbered% uncomment this for unnumbered level heads



\begin{document}

\title[Article Title]{Emergent Spacetime and Stress--Energy from a Minimal Disformal Scalar Field}

\author[1]{\fnm{Moeidh} \sur{Mana Hanash}\orcidd{0009-0002-8426-825X}}\email{engmoeidh85@gmail.com}

\affil[1]{\orgdiv{Independent researcher}}

\abstract{
%In this work a minimal one-field framework is developed. A real scalar $\Phi$ defines both the effective spacetime geometry and associated stress–energy. The construction is a rank one disformal transformation,
%\begin{equation}
%S=\int d^4x\Big[-\tfrac{1}{2}(\partial\Phi)^2 - V(\Phi) - \tfrac{\alpha}{4}%(\partial\Phi)^4\Big],
%\qquad
%g^{\rm eff}_{\mu\nu}=\eta_{\mu\nu}+\alpha\,\partial_\mu\Phi\,\partial_\nu\Phi, %\label{1}
%\end{equation}
%with closed form inverse and determinant. The field equation is quasilinear, strictly second order, and hyperbolic for $1+\alpha X>0$ with $X=\eta^{\mu\nu}\partial_\mu\Phi\,\partial_\nu\Phi$. The quartic gradient acts as a Skyrme like stabilizer: a covariant virial identity yields $E_{\mathcal G}=\mathcal T+3V$, showing that finite energy stationary solitons are compatible with Derrick’s theorem in $3{+}1$ dimensions. The energy functional is coercive and bounded below for $\alpha>0$ and $V\!\ge\!0$, and 1-D kink-type numerics confirm stable minima and scaling relations.\\
\begin{abstract}
A central question in fundamental physics is whether the observed spacetime geometry
and gravitational interaction can emerge from simpler microscopic degrees of freedom
without spoiling the empirical success of general relativity. Motivated by this, and by
the dark sector puzzle, I investigate a minimal ``proto--field'' scenario in which a
single real scalar field underlies both matter and geometry via a specific
rank-one \emph{disformal} deformation of the metric (an algebraic modification built
from first derivatives of the scalar).

The scalar sector is governed by a quartic $P(X)$ Lagrangian, which induces an
effective metric $g^{\rm eff}_{\mu\nu}=\eta_{\mu\nu}+\alpha\,\partial_\mu\Phi\,
\partial_\nu\Phi$ in a local inertial frame. In the low-energy effective action I add
a small Einstein--Hilbert term for this composite metric, treated as an effective
(ultimately loop-induced) Planck-scale coupling, thereby ensuring an exactly GR-like
Newtonian and post-Newtonian limit in the infrared. I show that the resulting one-field
theory is quasilinear, strictly second order and free of Ostrogradsky instabilities in
a well-defined causal (hyperbolic) regime, and that its quartic gradient term provides
a genuine Skyrme-like stiffness: a covariant virial identity $E_4 = E_2 + 3E_V$
guarantees the existence of finite-energy, linearly stable solitons in three spatial
dimensions. A full analysis of the conservative weak-field sector up to second
post-Newtonian order, in the standard parametrized post-Newtonian (PPN) framework,
yields $\gamma=\beta=1$ with no preferred-frame or non-conservative effects; model-specific
corrections from the quartic $P(X)$ term first appear only at 2.5PN (dissipative) and
3PN/4PN (conservative) order. These results demonstrate that a single disformal scalar
can underpin an emergent, GR-like metric theory whose classical solitonic sector is
nonlinearly stable and whose weak-field phenomenology passes all current tests.
\end{abstract}

\keywords{geometric theory of gravitation, minimal disformal scalar framework,
stress--energy, general relativity, scalar--tensor theories, post-Newtonian approximation}
}

\keywords{geometric theory of gravitation, minimal disformal scalar framework,
stress--energy, general relativity, scalar--tensor theories, post-Newtonian approximation}
%\\\emph{Scope of Part I.}
%A universal minimal coupling of all matter to $g^{\rm eff}_{\mu\nu}$ is assumed, with photons following the effective metric. Restricting to the \emph{conservative weak-field sector through second post-Newtonian order}, the parametrized post-Newtonian values $\gamma = \beta = 1$ are obtained with no preferred-frame effects, reproducing the general-relativistic results for light deflection and Shapiro delay. Dissipative dynamics (radiation reaction and waveform phasing), higher order PN contributions, and strong-field phenomenology are deferred to Parts II–III of the trilogy.

%\keywords{keyword1, Keyword2, Keyword3, Keyword4}

%%\pacs[JEL Classification]{D8, H51}

%%\pacs[MSC Classification]{35A01, 65L10, 65L12, 65L20, 65L70}

\maketitle

\newpage
%%%%%%%%%%%%%%%%%%%%%%%%%%%%%%%%%%%% Nomenclature
\begin{longtable}{@{}p{0.27\linewidth} p{0.70\linewidth}@{}}
\caption*{\Large Nomenclature}\\
\toprule
\textbf{Mathematical symbol} & \textbf{Description / meaning} \\
\midrule
\endfirsthead
\toprule
\textbf{Mathematical symbol} & \textbf{Description / meaning} \\
\midrule
\endhead
\bottomrule
\endfoot
\bottomrule
\endlastfoot
\multicolumn{2}{@{}l}{\textbf{A. Core fields and geometry}}\\[2pt]
$\Phi$ & Real scalar \emph{proto-field} that generates both effective geometry and stress–energy.\\
$\partial_\mu \Phi$ & Spacetime gradient of $\Phi$; the preferred direction entering the disformal update.\\
$\eta_{\mu\nu}$ & Local Minkowski metric used in weak-field expansions, signature $(-,+,+,+)$.\\
$g_{\mu\nu}$ & Dynamical spacetime metric seen by matter and waves (carries the EH seed and couples minimally to all fields).\\
$g^{\mu\nu}$ & Inverse of $g_{\mu\nu}$; reduces to $\eta^{\mu\nu}$ when the deformation vanishes.\\
$X \equiv g^{\mu\nu}\partial_\mu\Phi\,\partial_\nu\Phi$ & Covariant kinetic invariant of the proto-field (used in the $P(X)$ sector and health band).\\
$\alpha$ & Disformal coupling constant controlling the strength of the metric deformation (dimension of length$^{\mathbf{4}}$ in geometric units; equivalently mass$^{-4}$).\\
$1+\alpha X>0$ & Causal / hyperbolicity condition ensuring $g_{\mu\nu}$ is Lorentzian and evolution is well-posed.\\
$V(\Phi)$ & Self-interaction potential (generic; specific choices discussed where needed).\\
$\Phi_\infty$ & Vacuum/background value of $\Phi$ at an extremum of $V$.\\
$\phi$ & Small perturbation: $\Phi=\Phi_\infty+\phi$.\\
$m^2 \equiv V''(\Phi_\infty)$ & Mass squared of small fluctuations around the vacuum.\\
$G,\ c$ & Newton’s constant and speed of light (often $G=c=1$).\\

\addlinespace[4pt]
\multicolumn{2}{@{}l}{\textbf{B. Derivatives, operators, and index conventions}}\\[2pt]

$\partial_\mu$ & Partial derivative in coordinates $x^\mu$.\\
$\nabla_\mu$ & Covariant derivative compatible with $g_{\mu\nu}$.\\
$\square \equiv \eta^{\mu\nu}\partial_\mu\partial_\nu$ & Flat-space d’Alembertian.\\
$\square_g \equiv \nabla_\mu\nabla^\mu$ & Covariant wave operator defined by $g_{\mu\nu}$.\\
$\delta^\mu{}_\nu,\ \delta_{ij}$ & Kronecker deltas (spacetime / spatial).\\
$\epsilon_{ijk}$ & Levi–Civita symbol in 3D Euclidean space.\\
$(\cdots)_{,\mu}$ & Comma notation for partial derivative $\partial_\mu(\cdot)$.\\
$(\cdots)_{;\mu}$ & Semicolon notation for covariant derivative $\nabla_\mu(\cdot)$.\\

\addlinespace[4pt]
\multicolumn{2}{@{}l}{\textbf{C. Action, field equations, and stress--energy}}\\[2pt]

$S[\Phi]$ & Proto-field action (flat or curved index form, depending on context).\\
$T_{\mu\nu}$ & Stress–energy tensor of $\Phi$ (flat-metric form).\\
$T^{(g)}_{\mu\nu}$ & Stress–energy defined with respect to $g_{\mu\nu}$ (curved-metric form).\\
$T_{00}$ & Energy density in a chosen frame.\\
$T^i{}_j$ & Spatial stress tensor components; diagonal parts act as pressures.\\
$T \equiv T^\mu{}_\mu$ & Trace of the stress–energy tensor.\\
$\nabla_\mu T^{(g)\mu}{}_{\nu}=0$ & Covariant conservation with respect to $g_{\mu\nu}$.\\

\addlinespace[4pt]
\multicolumn{2}{@{}l}{\textbf{D. Static, spherical, and solitonic configurations}}\\[2pt]

$r$ & Areal radius (spherical symmetry).\\
$M(r)$ & Enclosed mass/energy within radius $r$ (as defined in the text).\\
$\rho(r)$ & Energy density profile for static solutions.\\
$p_r(r),\ p_t(r)$ & Radial and tangential pressures; anisotropy $\Delta \equiv p_t-p_r$.\\
$E_2$ & Quadratic (gradient/tension) energy contribution.\\
$E_4$ & Quartic (disformal/pressure) energy contribution.\\
$E_V$ & Potential energy contribution (from $V(\Phi)$).\\
$E_{\rm tot}$ & Total energy (sum of $E_2,E_4,E_V$).\\
$\mathcal{R}_\star$ & Characteristic radius (e.g., size of a solitonic core).\\
$\nu(r),\ \lambda(r)$ & Metric potentials in standard static-spherical line element if/when used.\\
$\sigma(r)$ & Generic profile function used in ansätze (if introduced).\\
$\Gamma$ & Adiabatic index for stability considerations (when applicable).\\

\addlinespace[4pt]
\multicolumn{2}{@{}l}{\textbf{E. Virial identity and stability}}\\[2pt]

$\mathcal{V}$ & Virial functional used to derive scale relations.\\
$E_4 = E_2 + 3E_V$ & Virial balance identity (schematic form; exact coefficients per derivation).\\
$\delta^2 E$ & Second variation of total energy (stability criterion).\\
$\omega^2$ & Squared frequency of small radial oscillations (stability if $\omega^2>0$).\\

\addlinespace[4pt]
\multicolumn{2}{@{}l}{\textbf{F. Weak-field and post-Newtonian (PN) sector}}\\[2pt]

$U$ & Newtonian gravitational potential ($\nabla^2 U = -4\pi G\rho$).\\
$V_i$ & Vector (gravitomagnetic) potential in PN expansion.\\
$x \equiv (GM\Omega/c^3)^{2/3} \simeq GM/(ac^2)$ & Standard PN frequency/velocity parameter (binary).\\
$M=M_1+M_2$ & Total mass of the binary system.\\
$\mu = M_1 M_2/M$ & Reduced mass.\\
$a$ & Binary separation (orbital semi-major axis or equivalent PN parameter).\\
$\Omega$ & Orbital angular frequency.\\
$g_{00},\,g_{0i},\,g_{ij}$ & Metric components through 2PN (match GR in conservative sector).\\
$\bar{\alpha} \equiv \alpha M^2/a^4$ & Dimensionless PN coupling built from $\alpha$ (scaling indicator).\\
$S_3 = U|\nabla U|^2$ & Minimal near-zone source generating 3PN conservative corrections.\\
$S_4 = |\nabla U|^4$ & Minimal near-zone source generating 4PN conservative corrections.\\
$C_{3}^{\rm ren},\,C_{4}^{\rm ren}$ & Renormalized outer-zone coefficients at 3PN/4PN extracted from numerical fits.\\
$R_\star$ & Outer-zone matching/extrapolation radius used to determine $C_{n}^{\rm ren}$.\\
$R_{\rm ring}$ & Ring-averaging radius used to suppress near-field contamination.\\
$L/a$ & Box-size to separation ratio in periodic Poisson solves.\\
$k=0$ mode & Zero (DC) Fourier mode removed to enforce zero-mean Poisson solutions.\\

\addlinespace[4pt]
\multicolumn{2}{@{}l}{\textbf{G. Gauge-invariant diagnostics and waveform phasing}}\\[2pt]

$\delta K(x)$ & Fractional periastron-advance shift (gauge-invariant conservative diagnostic).\\
$\delta z_1(x)$ & Detweiler redshift correction (gauge-invariant conservative diagnostic).\\
$F$ & Gravitational-wave energy flux (dissipative sector).\\
$\displaystyle \frac{dE}{dt}=-F$ & Energy balance law relating conservative energy and dissipative flux.\\
$\tilde{h}(f)$ & Frequency-domain strain; SPA amplitude $\propto f^{-7/6}$.\\
$\Psi(f)$ & Frequency-domain GW phase (SPA).\\
$v \equiv (\pi M f)^{1/3}$ & PN velocity parameter used in SPA phasing.\\
$\Delta\Psi_\alpha(f)$ & $\alpha$-sector correction to the GW phase.\\
$\chi_5,\ \chi^{\ln}_5$ & Dimensionless coefficients of the leading 2.5PN dissipative phase terms ($v^5$ and $v^5\ln v$).\\

\addlinespace[4pt]
\multicolumn{2}{@{}l}{\textbf{H. Observational \& data-analysis quantities}}\\[2pt]

O3, O4a & LVK observing runs used to constrain the coupling.\\
$\alpha_{\max}$ & Upper bound on $\alpha$ inferred from phase-posterior constraints.\\
PTA & Pulsar Timing Array (used for weak-field/low-frequency constraints).\\

\addlinespace[4pt]
\multicolumn{2}{@{}l}{\textbf{I. Notational and miscellaneous}}\\[2pt]

$\delta(\cdot)$ & Perturbation/variation operator (context-dependent).\\
$\langle\cdots\rangle$ & Spatial/temporal or ring average, as defined in the numerical pipeline.\\
$\mathcal{O}(\cdot)$ & Big-O order symbol in PN or asymptotic expansions.\\
$\sim,\ \simeq$ & Asymptotic/scaling equality in appropriate limits.\\
\end{longtable}

%%%%%%%%%%%%%%%%%%%%%%%%%%%%%%%%%%%%%%%%%%%%%


%%%%%%%%%%%%%%%%%%%%%%%%%%%%%%%%%%%%%%%%%%%%%%%%%%%%%%%%%%%%%%%%%%%%%%%%%%%%%%%%
\section{Introduction}

General relativity (GR) is the geometric theory of gravitation by Albert Einstein that couples curvature of spacetime with matter \cite{Einstein1916_GR}.
The Einstein field equations relate geometry, through the Einstein tensor, with the stress--energy tensor defining matter distribution \cite{Ryder2009_GR,Wald2010_GR}.
The field equations model relativistic phenomena in spherically (see \cite{Azam2016_EPJC,Mazharimousavi2017_IJMPD,Morales2018_EPJC,Raposo2019_PRD,Lin2021_PRD,Oleary2021_Springer,Vossos2023_AIP,Kumar2025_EPJC} and references therein) and cylindrically (see \cite{Azam2016_EPJC_Cyl,Yousaf2017_EPJC,Ahmed2018,Bronnikov2019_PRD,Rehman2025_EPJC} and references therein) symmetric regimes.
These models must satisfy several stability criteria, including the standard energy conditions, constraints on radial sound speed, adiabatic index, Herrera's cracking condition, Jeans stability analysis, and Tolman--Oppenheimer--Volkoff equation (see \cite{Kolassis1988,Hawking2023,Herrera1997_PhysRep,Brassel2021a,Brassel2021b,Herrera1992_PLA,adiabatic,Heintzmann1975,Hillebrandt1976,Chan1993,Oppenheimer1939,Azam2015} and references therein). Experimental tests of GR from equivalence principle to solar system experiments and gravitational radiation have confirmed the validity of GR in weak and strong field regimes \cite{Will2014_LivingRev,Dicke1959,Schiff1960}. The equivalence principle shows unity between inertia and gravitation, serving as a founding postulate in GR. Through this principle, the inertial properties of matter, shown by the stress--energy tensor, connect to spacetime curvature describing gravitation. A freely falling object moving through this curved spacetime follows a geodesic path, which represents its inertial motion in a gravitational field.

During the past century, numerous attempts have been made to extend or reinterpret Einstein's theory of gravitation. One early attempt was Kaluza's unification of gravity and electromagnetism in five-dimensional spacetime in 1921, later reinterpreted by Klein in 1926 \cite{Kaluza1921,Klein1926}. In the mid-20th century, Jordan introduced scalar–tensor models with a dynamic Newton's constant \cite{Jordan1955}. This led to the Brans--Dicke theory in 1961, where a scalar field coupled to curvature embodied Mach's principle \cite{Brans1961}. In 1970, Buchdahl generalized the Einstein--Hilbert action to non-linear $f(R)$ functions \cite{Buchdahl1970}, while Horndeski formulated the most general scalar–tensor action for second-order field equations \cite{Horndeski1974}. Lovelock established ghost-free higher-dimensional Einstein tensor extensions \cite{Lovelock1971}. Extra dimensional approaches later emerged in the Randall--Sundrum model \cite{Randall1999}. The Eddington-inspired Born--Infeld gravity by Bañados and Ferreira softened curvature singularities while maintaining GR properties \cite{Banados2010_EiBI}. Recent developments include massive gravity \cite{deRham2010}, teleparallel $f(T)$ gravity \cite{Cai2016_fT}, and non-metricity based $f(Q)$ gravity \cite{Jimenez2018_fQ}, offering new matter--geometry unification perspectives.

Minimalism has guided the search for fundamental physics, favoring theories with fewer assumptions and fields for their clarity and predictive power. Several minimalist extensions of GR reflect this approach, each with limitations. Scalar-tensor theories such as the Brans--Dicke \cite{Brans1961} theory, $f(R)$ gravity \cite{Buchdahl1970}, and Horndeski model \cite{Horndeski1974} add one scalar degree of freedom but rely on an independent metric. Teleparallel $f(T)$ gravity \cite{Cai2016_fT} modifies geometry, yet matter and spacetime remain separate. These models demonstrate the challenge of extending GR without extra assumptions or pathological terms.
The proto-field gravity model (PFGM) proposes a more radical form of minimalism, positing a
single real scalar field $\Phi$ as the \emph{only additional} dynamical degree of freedom
beyond the metric. Effective spacetime geometry and stress--energy are tied to gradients of
$\Phi$ via a rank-one disformal structure (see \cite{Bekenstein1993,Ip2015_DisformalPPN,Domenech2015,Tsujikawa2015,Bettoni2013,Chiba2020,Takahashi2022,BenAchour2025} and
references therein). The scalar sector is described by a $P(X)$ theory with a quartic
derivative term, and the metric carries an Einstein--Hilbert (EH)
term so that the infrared tensor sector reduces to GR. The scalar sector is described by a $P(X)$ theory with a quartic
derivative term, and the metric carries an Einstein--Hilbert (EH)
term so that the infrared tensor sector reduces to GR. In this work I treat
the corresponding Planck scale $M_{\rm ind}$ as an effective low-energy
parameter (an ``EH seed''). Its microscopic origin as a loop-induced
scale in the same proto--field disformal sector—where $g^{\rm eff}_{\mu\nu}$
acquires an Einstein--Hilbert kinetic term and a curvature--squared tail
via a standard heat-kernel expansion—is developed in a companion
analysis~\cite{Hanash_Induced_GR_and_EFT_2025}. For the classical and weak-field
results presented here, only the existence of such an EH term is required. In this combined system the equations remain quasilinear and second order, avoiding Ostrogradsky instabilities \cite{Woodard2015}. This one-field approach addresses GR's prediction of singularities at extreme densities by eliminating the separation between geometry and sources, creating feedback that can regulate high density states. As a flat-space prototype, the scalar sector can be written as a $P(X)$ theory with an
algebraic disformal map,
\begin{equation}
S=\int d^4x\Big[-\tfrac{1}{2}(\partial\Phi)^2 - V(\Phi) - \tfrac{\alpha}{4}(\partial\Phi)^4\Big],
\qquad
g^{\rm eff}_{\mu\nu}=\eta_{\mu\nu}+\alpha\,\partial_\mu\Phi\,\partial_\nu\Phi, \label{1}
\end{equation}
with closed form inverse and determinant. The theory is consistently defined within the hyperbolic domain (here and throughout the paper, the signatures $(-,+,+,+)$ and $X\equiv \eta^{\mu\nu}\partial_\mu\Phi\,\partial_\nu\Phi$ have been used; thus, for static fields, $X=|\nabla\Phi|^2$.)
\begin{equation}
1+\alpha X > 0,
\qquad
X \equiv \eta^{\mu\nu}\partial_\mu \Phi \,\partial_\nu \Phi , \label{2}
\end{equation}
where the Cauchy problem is well posed and the causal cones of the disformal metric govern signal propagation \cite{Bekenstein1993,Bettoni2013}. Equations~\eqref{1}--\eqref{2} should be understood as a flat-background prototype that illustrates the rank-one disformal structure and the quartic $P(X)$ sector. The full gravitational theory used in this paper adds an Einstein--Hilbert seed for the operational metric and promotes $X$ to $X=g^{\mu\nu}\partial_\mu\Phi\,\partial_\nu\Phi$; see Sec.~\ref{subsec:EH-seed}. This ensures a consistent causal structure and allows stability analyses within a mathematically controlled regime. Beyond formal consistency, PFGM demonstrates robust phenomenology. Sound speeds remain subluminal within a healthy band, virial identities hold across boundary conditions, nonlinear residuals decay rapidly under iteration, and kink-like solutions emerge with finite localized energy. These diagnostic results (see Sec.~D and App.~D) support PFGM as a viable, minimalist yet powerful unification: matter and geometry from a single field.

The universe provides motivation for PFGM through cosmological observations. Planck 2018 results \cite{Planck2018_Cosmology} show that dark matter and dark energy dominate the cosmic energy budget, with unknown physical origin. If a scalar field $\Phi$ underlies both spacetime geometry and the associated stress--energy, dark sector phenomena could be reformulated more economically, making PFGM a strong candidate for unification.

Quantum field considerations strengthen this case. With the vacuum expectation value of $\Phi$ defining the effective metric, radiative corrections gain geometric interpretation. Within the strong-coupling scale $\Lambda_\alpha=\alpha^{-1/4}$, higher-dimension operators are suppressed without ghosts \cite{Burgess2007_EFT}. The disformal transformation's algebraic form ensures consistency without hidden symmetries or fine-tuned cancellations.

The formulation is minimal: $\Phi$ defines measurement units, generates stress--energy, and produces metric curvature. Localized $\Phi$ configurations appear as finite-energy solitons shaping spacetime geometry. The quartic gradient term $\alpha (\partial \Phi)^4$ acts as internal pressure, modifying scaling behaviour. With virial balance, the energy functional reaches minimum at finite size, circumventing Derrick's theorem \cite{Derrick1964} while maintaining second-order dynamics.

This stabilizing mechanism parallels the Skyrme term in nuclear soliton models, where quartic interactions generate finite-size baryonic solitons \cite{Skyrme1961,Adkins1983}. The analogy shows stability achieved through the emergent metric's structure rather than new fields or exotic operators.

Earlier attempts to construct purely scalar gravity models faced two obstacles. Derrick's theorem rules out static scalar solitons in three spatial dimensions with only canonical kinetic and potential terms \cite{Derrick1964}. Any candidate configuration dilates without bound or collapses to zero size. Higher-derivative scalar theories suffer from Ostrogradsky instabilities, with Hamiltonians unbounded from below \cite{Woodard2015}. PFGM avoids both: the quartic correction arises algebraically from the disformal structure, maintaining quasilinear and second-order field equations with one healthy degree of freedom.

In modified gravity, PFGM is distinctive. Models such as the Brans--Dicke \cite{Brans1961}, Horndeski \cite{Horndeski1974}, and Galileon theories \cite{Nicolis2009_Galileon,Deffayet2011_Galileon} needed additional operators for stability. Mimetic \cite{Chamseddine2013_Mimetic} and Born--Infeld theories \cite{Banados2010_EiBI} explored nonlinear completions. Reviews are found in \cite{Clifton2012,Joyce2015,deRham2014_Review}. PFGM introduces no extra fields, hidden dimensions, or fine-tuned cancellations. Its stabilizer emerges from the disformal geometry defined by $\partial_\mu\Phi$, evading Derrick's theorem and Ostrogradsky instabilities while providing a basis for field theory and cosmology \cite{Mukhanov2005}.

This article focuses on the \emph{foundational and conservative weak-field
sector through second post-Newtonian (2PN) order}. Universal minimal coupling of matter
(including electromagnetism) to the dynamical metric $g_{\mu\nu}$ is assumed, and a small
Einstein--Hilbert seed is included in the action \eqref{eq:full_action}. Within the healthy band
$1+\alpha X>0,\;1+3\alpha X>0$ and the PN-suppressed regime $\alpha X\lesssim\mathcal{O}(\epsilon^2)$,
the conservative PPN parameters match GR through 2PN, with $\gamma=\beta=1$ and Solar-System
light-deflection and Shapiro delay identical to GR (Appendix~\ref{app:PN}). Radiative dynamics,
waveform phasing, and higher-order PN effects (3PN/4PN) are deferred to future work on PN waveform phasing and strong-field structure.

The remainder of this paper is organized as follows.
The disformal construction of the emergent metric, from which the field equations are derived, is introduced in Section~2.
In Section~3, the covariant formulation of PFGM is established, focusing on hyperbolicity, causal structure, and the absence of Ostrogradsky instabilities.
The analysis of the effective stress--energy tensor and stability bands is presented in Section~4, emphasizing the role of the quartic gradient term and its consistency within effective field theory.
A generalized virial identity is developed in Section~5; the model’s circumvention of Derrick’s theorem to permit finite--size solitons is presented.
Numerical kink solutions are presented in Section~6, providing explicit evidence for stability and finite localized energy.
In Section~7, weak-field and post-Newtonian limits are examined in relation to experimental tests of gravity.
Finally, concluding remarks and an outlook on future directions is offered in Section~8.

Clarifying how the present construction differs from established
approaches is crucial. In scalar--tensor and Horndeski theories, disformal metrics of the form
$g_{\mu\nu}=\eta_{\mu\nu}+\alpha\,\partial_\mu\phi\,\partial_\nu\phi$
have long been studied \cite{Bekenstein1993,Zumalacarregui2014,Bettoni2013,Tsujikawa2015},
but always in combination with an \emph{external} matter sector: the scalar modifies
gravity but does not itself supply stress--energy. Skyrme models, and their
self-gravitating Einstein--Skyrme extensions, stabilize solitons using quartic
gradient terms \cite{LuckockMoss1986_SkyrmeBH,Perapechka2017_Skyrmions}; however, these involve an ${\rm SU}(2)$ chiral multiplet and coexist with conventional
baryonic matter. Boson-star and Q-ball constructions show that complex scalars with
self-interaction can form long-lived lumps under gravity,\cite{Kaup1968_BosonStars,RuffiniBonazzola1969_BosonStars,LieblingPalenzuela2012_Review}
but rely on potential terms or quantum pressure rather than a single real field with
quartic gradients. By contrast, the PFGM welds the essential
ingredients into a one-field ontology, where the same scalar gradients (i) generate the
effective metric, (ii) provide the stress--energy, and (iii) furnish the quartic
pressure that stabilizes finite-size solitons. No external matter sector is needed.
This precise combination does not appear in the Horndeski, Skyrme, or boson-star
literatures.

Conservative weak field matches GR through 2PN ($\gamma=\beta=1$, no preferred–frame terms and here $\gamma,\beta$ denote the standard PPN parameters); the first deviation is dissipative at 2.5PN (flux sector), whereas conservative $\alpha$-effects
begin only at 3PN/4PN and are parametrically small on Solar–System/pulsar scales.

%%%%%%%%%%%%%%%%%%%%%%%%%%%%%%%%%%%%%%%%%%%%%%%%%%%%%%%%%%%%%%%%

\section{Minimal action, field equations, and emergent metric}
\label{sec:action}

The foundation of PFGM is a single real scalar field endowed with a quartic derivative interaction that simultaneously determines spacetime geometry and stress--energy. This section introduces the minimal Lorentz-invariant action, the emergent disformal metric, and the resulting field equations, formulated within a rigorously well-posed EFT framework.

\subsection{Postulates and notation}

Spacetime carries a dynamical Lorentzian metric $g_{\mu\nu}$ of signature $(-,+,+,+)$.
Greek indices $\mu,\nu,\dots$ run over $0,1,2,3$, and Latin indices $i,j,\dots$ denote spatial
components. Einstein summation is assumed. Covariant derivatives $\nabla_\mu$ are compatible
with $g_{\mu\nu}$, and the d’Alembertian is $\square_g \equiv g^{\mu\nu}\nabla_\mu\nabla_\nu$.
In the weak-field/post-Newtonian (PN) regime I often work in local coordinates where
$g_{\mu\nu}=\eta_{\mu\nu}+h_{\mu\nu}$ with $\eta_{\mu\nu}=\mathrm{diag}(-1,+1,+1,+1)$, but the
fundamental formulation is fully diffeomorphism invariant.

\paragraph{Conventions:}
\begin{equation}
X \;\equiv\; g^{\mu\nu}\,\partial_\mu\Phi\,\partial_\nu\Phi,
\qquad [\alpha]={\rm mass}^{-4},\qquad M_\alpha \equiv \alpha^{-1/4}. \label{new_1}
\end{equation}
The field $\Phi$ has canonical mass dimension one. In what follows, I use \emph{mass dimensions} when writing $[\alpha]$ (so $[\alpha]={\rm mass}^{-4}$); in length units with $c=\hbar=1$, this corresponds to $[\alpha]=L^{4}$. Unless stated otherwise, indices are raised and lowered with $g_{\mu\nu}$; in local inertial frames I set $g_{\mu\nu}\simeq\eta_{\mu\nu}$ for PN expansions.


The dynamical variable is a real scalar field $\Phi(x)$ with canonical mass dimension one. Its self-interaction potential $V(\Phi)$ has a nondegenerate vacuum at $\Phi_\infty$, with $V''(\Phi_\infty)>0$, excluding tachyonic modes. A positive constant $\alpha$ of mass dimension $-4$ controls the quartic derivative interaction and defines the EFT cutoff scale $\Lambda_\alpha=\alpha^{-1/4}$. $\alpha$ sets the scale for disformal corrections, constrained by Solar-System tests in analogy with disformal scalar–tensor theories subject to post-Newtonian bounds \cite{Ip2015_DisformalPPN,Burgess2020_EFTbook}.

The field equations are quasilinear and second order, ensuring well-posedness of the Cauchy problem: solutions exist, are unique, and depend continuously on initial data. Similar properties exist in broader scalar–tensor EFTs in singularity-avoiding coordinates \cite{Kovacs2020_WellPosed,AresteSalo2022_WellPosed}. This approach aligns with recent geometric formulations of scalar-field EFTs, treating field configuration space as metric geometry \cite{Cohen2025_Geometry}.

%%%%%%%%%%%%%%%%%%%%%%%%%%%%%%%%%%%%%%%%%%%%%%%%%%%%%%%%%%%%%%%

\subsection{Covariant augmented action with Einstein--Hilbert seed}
\label{subsec:EH-seed}

Guided by locality, diffeomorphism invariance, and the empirical success of GR in the weak
field, I take as fundamental a dynamical metric $g_{\mu\nu}$ and a single real scalar field
$\Phi(x)$ with action
\begin{equation}
S = \frac{M_{\rm ind}^2}{2}\int d^4x\,\sqrt{-g}\,R(g)
  + \int d^4x\,\sqrt{-g}\,\big[\,P(X) - V(\Phi)\,\big]
  + S_m[\psi,g],
\label{eq:full_action}
\end{equation}
where
\begin{equation}
X \equiv g^{\mu\nu}\partial_\mu\Phi\,\partial_\nu\Phi,
\qquad
P(X) = -\tfrac12 X - \tfrac{\alpha}{4}X^2,
\label{eq:P-of-X-def}
\end{equation}
and $S_m[\psi,g]$ denotes the matter action, minimally coupled to $g_{\mu\nu}$.
Here $M_{\rm ind}$ is an induced Planck scale in the effective-field-theory sense
(or, equivalently, an Einstein--Hilbert (EH) ``seed''), $V(\Phi)$ is a self-interaction
potential, and $\alpha>0$ has mass dimension $[-4]$ and sets the strength of the quartic gradient interaction.
A microscopic derivation of $M_{\rm ind}$ as a loop-induced scale in the proto--field sector
is given in the companion induced-gravity analysis~\cite{Hanash_Induced_GR_and_EFT_2025}, but is not
needed for the classical and weak-field results derived here. Matter fields and photons follow
geodesics of $g_{\mu\nu}$, so $g_{\mu\nu}$ is the operational metric seen by rods, clocks,
and light rays.
\footnote{I use the term ``induced'' in the EFT sense, i.e. $M_{\rm ind}$ may arise
from integrating out UV degrees of freedom; its microscopic derivation is not needed for the classical and weak-field results derived here and is addressed separately in Ref.~\cite{Hanash_Induced_GR_and_EFT_2025}.}

I work within the hyperbolic domain
\begin{equation}
1+\alpha X>0,
\end{equation}
and in the weak-field/PN sector in the PN-suppressed regime
\begin{equation}
\alpha X \lesssim \mathcal{O}(\epsilon^2),
\end{equation}
so that the scalar sector does not modify the metric at Newtonian or first post-Newtonian
order. In this regime the conservative PPN parameters match their GR values through second post-Newtonian order ($\gamma=\beta=1$; see Appendix~\ref{app:PN}).

Varying Eq.~\eqref{eq:full_action} with respect to $\Phi$ and $g^{\mu\nu}$ gives the coupled
field equations
\begin{align}
\nabla_\mu\big(P_X\,\nabla^\mu\Phi\big) - V'(\Phi) &= 0,
\label{eq:Phi-EOM-cov}\\[4pt]
M_{\rm ind}^2\,G_{\mu\nu}(g) &= T^{(\Phi)}_{\mu\nu} + T^{(m)}_{\mu\nu},
\label{eq:Einstein-EOM}
\end{align}
where $P_X\equiv \partial P/\partial X$, $T^{(m)}_{\mu\nu}$ is the matter stress tensor, and
the scalar stress tensor is
\begin{equation}
T^{(\Phi)}_{\mu\nu} = -2P_X\,\nabla_\mu\Phi\,\nabla_\nu\Phi
  + g_{\mu\nu}\big(P(X)-V(\Phi)\big).
\label{eq:Tphi-cov}
\end{equation}
Because $P(X)$ depends only on first derivatives of $\Phi$, the scalar equation is strictly
second order and avoids Ostrogradsky instabilities. The Bianchi identity implies
$\nabla^\mu T^{\rm (tot)}_{\mu\nu}=0$ with $T^{\rm (tot)}_{\mu\nu}
= T^{(\Phi)}_{\mu\nu}+T^{(m)}_{\mu\nu}$.

\subsection{Flat-space prototype and virial intuition}
\label{subsec:flat-prototype}

Many of the stability and virial arguments can be made transparent in a flat background.
Setting $g_{\mu\nu}=\eta_{\mu\nu}$ and neglecting backreaction on the metric, the scalar sector
reduces to a $P(X)$ theory in Minkowski space. In this flat prototype the action reads
\begin{equation}
S_{\rm flat}[\Phi] = \int d^4x \,
\left[-\tfrac{1}{2}\,\partial_\mu\Phi\,\partial^\mu\Phi - V(\Phi)
      - \tfrac{\alpha}{4}\big(\partial_\mu\Phi\,\partial^\mu\Phi\big)^2\right],
\label{eq:proto_action}
\end{equation}
with $\partial_\mu\Phi\,\partial^\mu\Phi=\eta^{\mu\nu}\partial_\mu\Phi\,\partial_\nu\Phi$. Varying the flat action \eqref{eq:proto_action} gives the Euler--Lagrange equation
\begin{equation}
\square \Phi - V'(\Phi) = 0,
\label{eq:EOM_basic}
\end{equation}
with $\square=\eta^{\mu\nu}\partial_\mu\partial_\nu$. The corresponding Hilbert stress--energy
tensor in flat space is
\begin{equation}
T_{\mu\nu} = \partial_\mu\Phi\,\partial_\nu\Phi
  - \eta_{\mu\nu}\left(\tfrac12\,\partial_\rho\Phi\,\partial^\rho\Phi + V(\Phi)\right).
\label{eq:Tmunu}
\end{equation}

The potential $V(\Phi)$ is assumed to possess at least one nondegenerate vacuum $\Phi_\infty$,
with $V'(\Phi_\infty)=0$ and $V''(\Phi_\infty)>0$, ensuring fluctuations propagate with real,
positive mass squared. A constant shift in $V(\Phi)$ is irrelevant in flat space but acts as
an effective cosmological constant in curved or emergent geometries.

The flat-space stress tensor and virial identities derived from Eq.~\eqref{eq:proto_action}
provide the intuition for the stabilized solitons discussed in Secs.~\ref{sec:Tmunu-stability}
and~\ref{sec:virial}; the fully covariant counterparts follow from
Eqs.~\eqref{eq:Phi-EOM-cov}--\eqref{eq:Tphi-cov} when $g_{\mu\nu}$ is dynamical.

%%%%%%%%%%%%%%%%%%%%%%%%%%%%%%%%%%%%%%%%%%%%%%%%%%%%%%%%%%%

\section{Emergent disformal metric}
\label{sec:disformal}

The central step in PFGM is to reinterpret scalar gradients as sources of stress–energy and spacetime geometry structure. Throughout this section I use $g^{\rm eff}_{\mu\nu}$ as a convenient algebraic representation of the $P(X)$ sector in a local inertial frame; the physical metric that carries the EH term and couples to matter is $g_{\mu\nu}$ in Eq.~\eqref{eq:full_action}. Unlike conventional scalar–tensor models with a postulated background metric and additional-operators-coupled scalar, here, the effective metric is induced algebraically by a single real scalar via a rank-one disformal update. This aligns with modern analyses of conformal and disformal transformations, including the DHOST classification controlling higher-derivative pathologies \cite{Zumalacarregui2014,Langlois2019}. In cosmology, observables are often invariant under disformal maps \cite{Domenech2015,Tsujikawa2015,Chiba2020}, showing when metric redefinitions are purely kinematical versus physically consequential. Since GW170817, near-luminal gravitational wave propagation has imposed sharp priors on modified-gravity sectors \cite{Baker2017,Ezquiaga2017}; within the hyperbolic domain defined below, the PFGM automatically respects these bounds.

\subsection{Field variation and quasilinear second-order character}
\label{sec:eom-secondorder}

It is not a priori obvious that varying the curved-variable action
\begin{equation}
S_\Phi \;=\; \int d^4x\,\sqrt{-g_{\mathrm{eff}}}\left[-\tfrac{1}{2}(g^{\mathrm{eff}})^{\mu\nu}
\partial_\mu\Phi\,\partial_\nu\Phi \;-\; U(\Phi)\right],
\qquad g^{\mathrm{eff}}_{\mu\nu}=\eta_{\mu\nu}+\alpha\,\partial_\mu\Phi\,\partial_\nu\Phi,
\label{eq:S-curved}
\end{equation}
will yield \emph{second-order} equations, because $g^{\mathrm{eff}}_{\mu\nu}$ depends on $\partial\Phi$.
Define the rank-one objects (using Eqs.~\eqref{eq:geff_inv}–\eqref{eq:geff_det})
\begin{equation}
A^{\mu\nu} \;\equiv\; \sqrt{-g_{\mathrm{eff}}}\,(g^{\mathrm{eff}})^{\mu\nu}
\;=\; \sqrt{1+\alpha X}\;\eta^{\mu\nu}
\;-\;\frac{\alpha}{\sqrt{\,1+\alpha X\,}}\;\partial^\mu\Phi\,\partial^\nu\Phi,
\qquad X=\eta^{\rho\sigma}\partial_\rho\Phi\,\partial_\sigma\Phi.
\label{eq:Amunu-def}
\end{equation}
Varying action in Eq. \eqref{eq:S-curved} with respect to\ $\Phi$, integrating by parts once, and collecting terms gives the divergence form
\begin{equation}
\partial_\mu\!\left( A^{\mu\nu}\,\partial_\nu\Phi \right) \;-\; \sqrt{-g_{\mathrm{eff}}}\,U'(\Phi) \;=\; 0.
\label{eq:EL-divergence}
\end{equation}
Expanding the first term shows explicitly that no third derivatives appear:
\begin{align}
\partial_\mu\!\left( A^{\mu\nu}\,\partial_\nu\Phi \right)
&= \underbrace{A^{\mu\nu}\,\partial_\mu\partial_\nu\Phi}_{\text{principal part}}
\;+\; \underbrace{(\partial_\mu A^{\mu\nu})\,\partial_\nu\Phi}_{\text{at most }\partial\partial\Phi}\!,
\label{eq:split}
\end{align}
where $\partial_\mu A^{\mu\nu}$ depends on $\partial_\mu X$ and hence on at most \(\partial\partial\Phi\):
\[
\partial_\mu X \;=\; 2\,\partial^\rho\Phi\,\partial_\mu\partial_\rho\Phi,
\qquad
\partial_\mu\!\left( \frac{1}{\sqrt{1+\alpha X}} \right)
= -\frac{\alpha}{2(1+\alpha X)^{3/2}}\,\partial_\mu X.
\]
Substituting these into Eq. \eqref{eq:split} shows that both terms contain at most \(\partial\partial\Phi\); any putative third derivatives cancel \emph{before} integration by parts because $A^{\mu\nu}$ depends only on \(\partial\Phi\). The principal symbol of Eq. \eqref{eq:EL-divergence} is therefore
\begin{equation}
\mathcal{P}^{\mu\nu}(\partial\Phi)\;\equiv\;A^{\mu\nu}
\;=\; \sqrt{1+\alpha X}\;\eta^{\mu\nu}
\;-\;\frac{\alpha}{\sqrt{\,1+\alpha X\,}}\;\partial^\mu\Phi\,\partial^\nu\Phi.
\label{eq:principal-symbol}
\end{equation}
The equation is \emph{quasilinear}, second order. Hyperbolicity follows when the effective light cone is Lorentzian, i.e., \ $1+\alpha X>0$ (Eq.~\eqref{eq:lorentz_cond}), which makes $\mathcal{P}^{\mu\nu}$ of signature $(-,+,+,+)$.\footnote{Equivalently, upon multiplying Eq. \eqref{eq:EL-divergence} by $(1+\alpha X)^{1/2}$ and using the identities Eq. \eqref{eq:raise_lower}–\eqref{eq:contraction}, the principal part reduces to a linear combination of $\Box\Phi$ and $(\partial\partial\Phi)$ projected along $\partial^\mu\Phi$, with coefficients depending only on $X$.}

\subsection{Degeneracy (DHOST) viewpoint and absence of extra modes}
\label{sec:DHOST}

Eq.~\eqref{eq:EL-divergence} belongs to a degenerate class familiar from disformal/Horndeski/DHOST constructions: although $g^{\mathrm{eff}}_{\mu\nu}$ depends on $\partial\Phi$, the Lagrangian contains \emph{no} second derivatives of $\Phi$ and thus avoids Ostrogradsky instabilities by construction. In the flat background used here, the action is exactly of $P(X)$ type:
\begin{equation}
\mathcal{L}_\Phi \;=\; P(X)-U(\Phi),
\qquad
P(X) \;=\; -\tfrac12 X \;-\; \tfrac{\alpha}{4}X^2,
\label{eq:PX-form}
\end{equation}
which is known to yield second-order Euler--Lagrange equations. The curved-variable rewriting Eq. \eqref{eq:S-curved} is related to Eq. \eqref{eq:PX-form} by an \emph{invertible} rank-one disformal map (invertible precisely when $1+\alpha X>0$); thus, no additional degree of freedom is introduced.\footnote{For background on disformal transformations and degeneracy conditions that eliminate higher-derivative ghosts, see e.g., \ Bekenstein (1993); Zumalacárregui \& García-Bellido (2014); Langlois \& Noui (2016); Kobayashi (2019).}
Conditions that ensure a healthy Cauchy problem (no ghost, no gradient instability) in the small-amplitude regime, for our sign convention $(-+++)$, follow from the quadratic-action coefficients $K_t=-(P_X+2X P_{XX})$ and $K_s=-P_X$:
\begin{equation}
P_X \;<\; 0,
\qquad
P_X+2X P_{XX} \;<\; 0,
\qquad
\big(P_X=-\tfrac12(1+\alpha X),\; P_{XX}=-\tfrac{\alpha}{2}\big),
\label{eq:no-ghost-no-gradient}
\end{equation}
equivalently, for $P(X)=-\tfrac12 X-\tfrac{\alpha}{4}X^2$, the conditions are
$1+\alpha X>0$ and $1+3\alpha X>0$, which is used in Sec.~\ref{sec:Tmunu-stability}.

\subsection{Gradient-defined metric}

Because the canonical stress--energy is quadratic in field gradients, it is natural to view
$\partial_\mu\Phi$ as defining a preferred direction. In a local inertial frame, the $P(X)$
sector can be encoded in an emergent rank-one disformal metric, Eq.~\eqref{eq:geff_def},
with $\alpha>0$ of mass dimension $[-4]$, $[\partial]=1$ and $[\Phi]=1$, I have $[X]=4$; the dimensionless combination is $\alpha X$ (since $[\alpha]=-4$). Eq.~\eqref{new_1} and Eq.~\eqref{eq:geff_def} are therefore dimensionally consistent. In the full EH-seeded theory, this rank-one structure should be understood as the local representation of the scalar sector; the dynamical metric entering the Einstein equations and matter coupling is $g_{\mu\nu}$.


As a rank-one update of $\eta_{\mu\nu}$, Eq.~\eqref{eq:geff_def} admits exact algebraic inversion via the Sherman–Morrison identity Eq.~\eqref{eq:geff_inv}-\eqref{eq:geff_det}.
A direct contraction verifies $(g^{\mathrm{eff}})^{\mu\rho}g^{\mathrm{eff}}_{\rho\nu}=\delta^\mu_\nu$. Thus all geometric objects remain algebraic in $\partial\Phi$, and no additional propagating degrees of freedom appear.

Lorentzian signature is maintained if
\begin{equation}
1+\alpha X > 0 .
\label{eq:lorentz_cond}
\end{equation}
With the $(-+++)$ convention,
\begin{equation}
X = -(\partial_t\Phi)^2 + |\nabla\Phi|^2 .
\label{eq:Xexplicit}
\end{equation}
Large time-like gradients must satisfy $\alpha\dot\Phi^2<1$ to preserve signature. The eigenvalue structure makes this explicit: three eigenvalues of $g^{\mathrm{eff}}_{\mu\nu}$ coincide with those of $\eta_{\mu\nu}$, while the fourth is rescaled by $1+\alpha X$.


Within the domain Eq.~\eqref{eq:lorentz_cond}, the characteristics of the scalar sector follow the
null cones of $g^{\mathrm{eff}}_{\mu\nu}$. In the EH-seeded formulation, however, the physical line
element for rods and clocks is
\begin{equation}
ds^2 = g_{\mu\nu}\,dx^\mu dx^\nu ,
\label{eq:line_element}
\end{equation}
with universal minimal coupling of matter to $g_{\mu\nu}$ via Eq.~\eqref{eq:full_action}. The
emergent metric $g^{\mathrm{eff}}_{\mu\nu}$ is best viewed as a convenient algebraic representation
of the $P(X)$ sector in a local inertial frame, whose cones lie inside (or coincide with) the matter
cones defined by $g_{\mu\nu}$ on the healthy branch.



Two algebraic relations recur in subsequent derivations:
\begin{align}
(g^{\mathrm{eff}})^{\mu\nu}\partial_\nu \Phi &= \frac{\partial^\mu \Phi}{1+\alpha X},
\label{eq:raise_lower}\\
(g^{\mathrm{eff}})^{\mu\nu}\partial_\mu\Phi\,\partial_\nu\Phi &= \frac{X}{1+\alpha X}.
\label{eq:contraction}
\end{align}
They guarantee that the equations of motion derived from $g^{\mathrm{eff}}_{\mu\nu}$ contain at most second derivatives of $\Phi$.

The scalar action can be recast as
\begin{equation}
S_\Phi = \int d^4x\,\sqrt{-g_{\mathrm{eff}}}
\left[-\tfrac{1}{2}(g^{\mathrm{eff}})^{\mu\nu}\partial_\mu\Phi\,\partial_\nu\Phi - V(\Phi)\right],
\label{eq:curved_action}
\end{equation}
where $V(\Phi)$ is the potential. Varying Eq. \eqref{eq:curved_action} gives
\begin{equation}
\frac{1}{\sqrt{-g_{\mathrm{eff}}}}
\partial_\mu\!\left(\sqrt{-g_{\mathrm{eff}}}\,(g^{\mathrm{eff}})^{\mu\nu}\partial_\nu\Phi\right)
- V'(\Phi) = 0 ,
\label{eq:field_eq}
\end{equation}
which is quasilinear and second order. The characteristics of Eq.~\eqref{eq:field_eq} coincide with the null cones of $g^{\mathrm{eff}}_{\mu\nu}$, consistent with well-posedness analyses of scalar–tensor EFTs.


For $|\alpha X|\ll1$, Eqs.~\eqref{eq:geff_inv}–\eqref{eq:geff_det} expand as
\begin{align}
(g^{\mathrm{eff}})^{\mu\nu} &= \eta^{\mu\nu} - \alpha\,\partial^\mu\Phi\,\partial^\nu\Phi + \mathcal{O}((\alpha X)^2),
\label{eq:inv_small}\\[4pt]
\sqrt{-g_{\mathrm{eff}}} &= 1 + \tfrac{1}{2}\alpha X + \tfrac{1}{8}(\alpha X)^2 + \cdots ,
\label{eq:det_small}
\end{align}
making explicit that disformal corrections enter perturbatively at $\mathcal{O}(\alpha X)$. This expansion underlies post-Newtonian bookkeeping and EFT power counting \cite{Burgess2007_EFT,Burgess2020_EFTbook}.

%%%%%%%%%%%%%%%%%%%%%%%%%%%%%%%%%%%%%%%%%%%%%%%%%%%%%%%%%%%%%

\subsection{Relation to Horndeski, mimetic, and $k$-essence frameworks}

Disformal metrics of the form
\begin{equation}
g_{\mu\nu} = A(\Phi, X)\,\tilde{g}_{\mu\nu} + B(\Phi, X)\,\partial_\mu \Phi \,\partial_\nu \Phi ,
\label{eq:Bekenstein}
\end{equation}
were first introduced by Bekenstein \cite{Bekenstein1993}. The resulting second-order dynamics are encompassed within Horndeski theory and its degenerate extensions \cite{Horndeski1974,Zumalacarregui2014,Kobayashi2019}. In conventional Horndeski models, $\Phi$ and $g_{\mu\nu}$ are independent dynamical variables, with the action built from curvature invariants and explicit second derivatives of $\Phi$. By contrast, in PFGM, the effective metric arises algebraically from $\Phi$ itself, and after a field redefinition, the dynamics reduce to a canonical Klein--Gordon form with an algebraic potential.

A closer analogue is \emph{mimetic gravity}, in which the physical metric is obtained via the singular constraint
\begin{equation}
g^{\mu\nu}\partial_\mu \Phi \,\partial_\nu \Phi = -1 ,
\label{eq:MimeticConstraint}
\end{equation}
enforced by a Lagrange multiplier \cite{Sebastiani2017}. This prescription introduces an additional pressureless component, interpreted as dark matter. PFGM differs in two ways: (i) the deformation is non-singular, requiring only hyperbolicity condition $1+\alpha X>0$, with no auxiliary multiplier; and (ii) the same scalar generates both the effective geometry and stress--energy through its quartic gradient term, eliminating need for a separate dust component.
Non-canonical kinetic models in \emph{$k$-essence} use a general function $P(X,\Phi)$ to drive cosmic acceleration, but involve variable sound speeds and instabilities. The PFGM metric in Eq. \ref{eq:geff_def} resembles these, but algebraically ties the quartic derivative term to disformal structure. The scalar sector remains canonical and propagates one healthy scalar mode. In the EH-seeded version considered here, the metric carries the usual Einstein--Hilbert kinetic term; induced gravity and loop-generated curvature operators will be analyzed in detail in future work. While Horndeski and mimetic models achieve stability through expanded operators or constraints, and $k$-essence modifies kinetic function, PFGM derives stability from geometry induced by $\partial_\mu\Phi$. This creates a minimal framework: a single scalar degree of freedom whose gradients generate effective curvature and stress--energy while maintaining second-order dynamics and energy positivity.
%%%%%%%%%%%%%%%%%%%%%%%%%%%%%%%%%%%%%%%%%%%%%%%%%%%%%%

\subsection{Geodesic motion of test clocks and rods}

In a one–field ontology, any operational device that measures space or time must itself be
composed of localized excitations of $\Phi$. Finite–energy configurations---kinks in one
dimension, vortices in two, and solitonic ``bags'' in three---can therefore serve as test
bodies \cite{MantonSutcliffe2004}. When their size is small compared with the scale over
which the background varies, the trajectory of the soliton’s center of energy extremizes the
proper time defined by the spacetime metric:
\begin{equation}
d\tau^2 = - g_{\mu\nu}(x)\,dx^\mu dx^\nu .
\label{eq:proper_time}
\end{equation}

Approximating the soliton as a point particle of fixed rest mass $m_0$, its leading–order action is
\begin{equation}
S_{\rm sol} = - m_0 \int d\lambda \,
\sqrt{- g_{\mu\nu}(x)\,\dot{x}^\mu \dot{x}^\nu} ,
\label{eq:soliton_action}
\end{equation}
where $x^\mu(\lambda)$ is parametrized by an affine parameter $\lambda$. Varying Eq. \eqref{eq:soliton_action} gives the geodesic equation
\begin{equation}
\ddot{x}^\mu + \Gamma^\mu_{\rho\sigma}(g) \, \dot{x}^\rho \dot{x}^\sigma = 0 ,
\label{eq:geodesic_equation}
\end{equation}
with Christoffel symbols constructed from $g_{\mu\nu}$. Thus, compact probes of $\Phi$ follow
geodesics of the physical spacetime metric, confirming the internal consistency of the construction.
Because matter couples minimally to $g_{\mu\nu}$ in Eq.~\eqref{eq:full_action}, finite-energy
excitations of the proto-field that I interpret as ``matter'' follow geodesics of $g_{\mu\nu}$; the
rank-one metric $g^{\rm eff}_{\mu\nu}$ continues to govern the scalar’s characteristic cones as
described in Sec.~\ref{sec:disformal}.

Because the disformal deformation depends only on first derivatives of $\Phi$, curvature tensors arise only at second order in gradients. In the weak–field limit, this reproduces Newtonian gravity, while post–Newtonian corrections appear at $\mathcal{O}(\alpha |\nabla \Phi|^2)$. A detailed parametrized post–Newtonian analysis is deferred to Sec.~7.
%%%%%%%%%%%%%%%%%%%%%%%%%%%%%%%%%%%%%%%%%%%%%%%%%%%%%%

\subsection{Field equations in curved indices}

The flat–space Euler--Lagrange Eq. \eqref{eq:EOM_basic} can be recast in covariant form once the disformal metric $g^{\rm eff}_{\mu\nu}$ is adopted as the fundamental variable. The curved–space action is
\begin{equation}
S = \int d^4x\,\sqrt{-g_{\rm eff}} \left[
\frac{1}{2(1+\alpha X)^{1/2}}\,g_{\rm eff}^{\mu\nu}\nabla_\mu \Phi \,\nabla_\nu \Phi
- \frac{V(\Phi)}{2(1+\alpha X)^{1/2}}
+ \frac{\alpha X^2}{2(1+\alpha X)^{3/2}}
\right],
\label{eq:curved_action_explicit}
\end{equation}
with $X=\eta^{\rho\sigma}\partial_\rho\Phi\,\partial_\sigma\Phi$.

Variation with respect to $\Phi$ gives
\begin{equation}
\nabla_\mu \!\left[
\frac{g_{\rm eff}^{\mu\nu}\nabla_\nu \Phi}{\sqrt{1+\alpha X}}
\right]
= \frac{V'(\Phi)}{\sqrt{1+\alpha X}}
+ \frac{\alpha}{(1+\alpha X)^{3/2}}
\,\nabla_\mu \Phi \,\nabla_\nu \Phi \,\nabla^\mu \nabla^\nu \Phi ,
\label{eq:curved_EOM}
\end{equation}
where $\nabla_\mu$ is the covariant derivative with respect to $g^{\rm eff}_{\mu\nu}$. All the terms involve at most two derivatives of $\Phi$; thus, Ostrogradsky instabilities are absent. In the weak–gradient limit $\alpha X\ll1$, Eq.~\eqref{eq:curved_EOM} reduces to the canonical Klein--Gordon equation.

The stress--energy tensor obtained by varying Eq. \eqref{eq:curved_action_explicit} with respect to $g^{\rm eff}_{\mu\nu}$ is
\begin{equation}
T^{(g)}_{\mu\nu} =
\frac{\nabla_\mu \Phi \,\nabla_\nu \Phi}{\sqrt{1+\alpha X}}
- g^{\rm eff}_{\mu\nu}\left[
\frac{\nabla_\rho \Phi \,\nabla^\rho \Phi}{2\sqrt{1+\alpha X}}
- \frac{V(\Phi)}{\sqrt{1+\alpha X}}
+ \frac{\alpha X^2}{2(1+\alpha X)^{3/2}}
\right].
\label{eq:T_curved}
\end{equation}
Because $X$ contains only first derivatives, Eq.~\eqref{eq:T_curved} is quadratic in gradients and reduces smoothly to the canonical Klein--Gordon tensor when $\alpha X\ll1$.

Finally, covariant conservation
\begin{equation}
\nabla^\mu T^{(g)}_{\mu\nu}=0 ,
\label{eq:T_conservation}
\end{equation}
holds identically as a consequence of Eq. \eqref{eq:curved_EOM}, consistent with the Bianchi identities of $g^{\rm eff}_{\mu\nu}$.

In summary, Eqs.~\eqref{eq:curved_EOM}--\eqref{eq:T_conservation} express proto–field gravity in manifestly covariant form: the scalar obeys a strictly second–order equation, its gradients generate the effective metric, and conservation follows from geometric consistency.
%%%%%%%%%%%%%%%%%%%%%%%%%%%%%%%%%%%%%%%%%%%%%%%%%%%%%%%

\subsection{Hyperbolicity and causal cones}

The kinetic operator for small fluctuations
$\varphi \equiv \Phi - \Phi_{0}$
around a background $\Phi_{0}(x)$ follows from expanding the covariant action Eq. \eqref{eq:curved_action_explicit} to quadratic order:
\begin{equation}
P[\varphi] = g^{\rm eff}_{\mu\nu}(\Phi_{0})\,\partial^\mu\varphi\,\partial^\nu\varphi + \cdots,
\qquad
g^{\rm eff}_{\mu\nu} = \eta_{\mu\nu} + \alpha\,\partial_\mu\Phi_{0}\,\partial_\nu\Phi_{0}.
\label{eq:fluctuation_metric}
\end{equation}
Hence, hyperbolicity and causal propagation are governed by the emergent metric $g^{\rm eff}_{\mu\nu}$.

\paragraph*{Inverse, determinant, and signature:}
For
\begin{equation}
X \equiv \eta^{\rho\sigma}\,\partial_\rho\Phi_{0}\,\partial_\sigma\Phi_{0},
\label{eq:X_background}
\end{equation}
the Sherman--Morrison identity gives
\begin{align}
(g^{\rm eff})^{\mu\nu}
&= \eta^{\mu\nu} - \frac{\alpha}{1+\alpha X}\,\partial^\mu\Phi_{0}\,\partial^\nu\Phi_{0},
\label{eq:inverse_background}\\
\det g^{\rm eff}_{\mu\nu} &= \det(\eta_{\mu\nu})\,(1+\alpha X),
\qquad
\sqrt{-g^{\rm eff}} = \sqrt{-\det\eta}\,\sqrt{1+\alpha X}.
\label{eq:det_background}
\end{align}
The metric remains Lorentzian ($-+++$) iff
\begin{equation}
1 + \alpha X > 0 .
\label{eq:hyperbolicity_condition}
\end{equation}


For $X>0$ (spacelike gradients), $1+\alpha X>1$ and the causal cone is narrowed relative to Minkowski.
For $X<0$ (timelike), the cone widens but stays Lorentzian if $1+\alpha X>0$. For $-1/3<\alpha X<0$ one has $c_s^2>1$ (hyperbolic but superluminal). In this work I restrict to $\alpha X\ge 0$, so $0<c_s^2\le 1$ throughout the configurations studied.


For static field configurations, $X=|\nabla\Phi_{0}|^{2}\ge 0$. Introducing the bookkeeping parameter $\varepsilon \equiv \alpha |\nabla\Phi_{0}|^{2}$, a conservative domain $\varepsilon<1/3$ implies
\begin{equation}
1+\alpha X = 1+\varepsilon > \tfrac{1}{3}.
\label{eq:static_hyperbolicity}
\end{equation}
The emergent causal cones lie strictly inside the Minkowski cone.


If $|\partial_{t}\Phi_{0}| \ll |\nabla\Phi_{0}|$, then
\begin{equation}
X = |\nabla\Phi_{0}|^{2} - (\partial_{t}\Phi_{0})^{2},
\label{eq:X_timevarying}
\end{equation}
and the correction to Eq. \eqref{eq:hyperbolicity_condition} is $\mathcal{O}(v^{2}/c^{2})$, well within the conservative EFT bound $\varepsilon<1/3$.


For the analytic kink solution in Sec.~6, the following is noted:
\begin{equation}
\omega^{2} = c_{\rm eff}^{2}(x)\,k^{2},
\qquad
c_{\rm eff}^{2}(x) = \frac{1}{1+\alpha |\nabla\Phi_{0}(x)|^{2}} < 1 .
\label{eq:dispersion_local}
\end{equation}
Fluctuations propagate strictly subluminally with respect to the emergent cones \cite{Babichev2008}.

\paragraph*{Summary:}
The PFGM is causal and well posed, provided
\begin{equation}
\alpha(\partial\Phi_{0})^{2} < 1,
\qquad
1+\alpha X > 0,
\qquad
\alpha |\nabla\Phi_{0}|^{2} < \tfrac{1}{3},
\label{eq:causal_summary}
\end{equation}
which ensure hyperbolicity and exclude gradient instabilities.
All the explicit backgrounds and numerical solutions presented in this work satisfy
Eq.~\eqref{eq:causal_summary}, typically within the conservative range
$\alpha(\partial\Phi_{0})^{2}\lesssim 0.1$. Approaching the degeneracy limit signals breakdown of the EFT description \cite{PapalloReall2017}. 

\paragraph*{Tensor sector and scope of this work.}
With the Einstein–Hilbert seed in Eq.~\eqref{eq:full_action}, the metric $g_{\mu\nu}$ carries
the usual massless spin–2 tensor degrees of freedom. In this work I simply assume such a
term is present in the low-energy effective action; its possible origin from one-loop
induction on the proto-field geometry is left for future work. In the infrared and in the weak-field regime, the propagation speed and polarizations of gravitational waves therefore coincide with those of GR, while the proto-field contributes a single scalar mode that is PN-suppressed in the healthy band. This work is restricted to the conservative weak-field sector; tensor radiation, waveform phasing, and the detailed 3PN/4PN conservative corrections are developed in future work.

%%%%%%%%%%%%%%%%%%%%%%%%%%%%%%%%%%%%%%%%%%%%%%%%%%%%%%%%%%%

\section{Stress--energy, stability, and hyperbolicity}
\label{sec:Tmunu-stability}
The next step is to examine the stress--energy content of the proto-field, because stability and causal well-posedness ultimately depend on the positivity of the Hamiltonian. With an ADM-like split, the individual roles of the canonical gradient, potential, and quartic derivative terms become explicit, allowing the establishment of both energy conditions and hyperbolicity margins.
\subsection{ADM-like split of the energy}
\label{subsec:ADMsplit}

Starting from the flat-space action with quartic derivative interaction, the following is obtained:
\begin{equation}
S[\Phi] = \int d^4x \left[-\tfrac12 (\partial\Phi)^2 - V(\Phi) - \tfrac{\alpha}{4}\,(\partial\Phi)^4\right],
\qquad \alpha>0,\; V(\Phi)\ge0.
\label{eq:flat_action_quartic}
\end{equation}
The Hamiltonian density is $\mathcal{H}=T_{00}$, and the total (conserved) energy on a constant–time slice $\Sigma\simeq\mathbb{R}^3$ is
\begin{equation}
E[\Phi] \equiv \int_{\Sigma}\! d^3x \,\mathcal{H}
= \int_{\Sigma}\! d^3x \left[\tfrac12\,(\partial_t\Phi)^2 + \tfrac12\,|\nabla\Phi|^2
+ V(\Phi) + \tfrac{\alpha}{4}\, \big((\partial\Phi)^2\big)^2 \right],
\label{eq:total_energy_def}
\end{equation}
with $(\partial\Phi)^2=-\dot\Phi^{\,2}+|\nabla\Phi|^2$. For static configurations ($\dot\Phi=0$) it is convenient to decompose
\begin{equation}
T \equiv \tfrac12\!\int_{\Sigma} |\nabla\Phi|^2\, d^3x, \qquad
E_V \equiv \int_{\Sigma} V(\Phi)\, d^3x, \qquad
E_4 \equiv \tfrac{\alpha}{4}\!\int_{\Sigma}\! |\nabla\Phi|^4\, d^3x,
\end{equation}
such that $E = T + E_V + E_4$.

Under the Derrick rescaling $\Phi(\mathbf{x})\mapsto \Phi(\lambda\mathbf{x})$, in $3$ spatial dimensions,
\begin{equation}
T(\lambda)=\lambda^{-1} T,\qquad
E_V(\lambda)=\lambda^{-3} E_V,\qquad
E_4(\lambda)=\lambda^{+1} E_4,
\label{eq:scalings}
\end{equation}
is consistent with the exponents $(-1,-3,+1)$ presented in Table~\ref{tab:scaling}. Because $\alpha>0$ and $V\ge0$, all three contributions are nonnegative, and $E[\Phi]$ is bounded from below.

\begin{table}[h]
\caption{Scaling behavior of the three positive contributions to $E$ in $3$ spatial dimensions under $\Phi(\mathbf{x})\!\to\!\Phi(\lambda\mathbf{x})$. The opposing exponents $-1$ and $+1$ for $T$ and $E_4$ ensure a finite-size minimum once $V$ is included.}
\label{tab:scaling}
\begin{tabular}{lcc}
\hline\hline
Component & Definition (static) & Scaling $x^i\!\to\!\lambda x^i$ \\
\hline
$T$ & $\tfrac12\!\int |\nabla\Phi|^2 d^3x$ & $\lambda^{-1}$ \\
$V$ & $\int V(\Phi) d^3x$ & $\lambda^{-3}$ \\
$E_4$ & $\tfrac{\alpha}{4}\!\int |\nabla\Phi|^4 d^3x$ & $\lambda^{+1}$ \\
\hline\hline
\end{tabular}
\end{table}
From the Lagrangian density
\begin{equation}
\mathcal{L} = -\tfrac12(\partial\Phi)^2 - V(\Phi) - \tfrac{\alpha}{4}\big[(\partial\Phi)^2\big]^2,
\label{eq:lagrangian_energy}
\end{equation}
the flat-space stress tensor is
\begin{equation}
T_{\mu\nu} = \Big[1+\alpha(\partial\Phi)^2\Big]\partial_\mu\Phi\,\partial_\nu\Phi
- \eta_{\mu\nu}\left(\tfrac12 (\partial\Phi)^2 + V(\Phi) + \tfrac{\alpha}{4}\big[(\partial\Phi)^2\big]^2\right).
\label{eq:stress_flat_quartic}
\end{equation}
Decompose $\partial_\mu\Phi = -(t\!\cdot\!\partial\Phi)\,t_\mu + q_\mu$ with $t^\mu t_\mu=-1$ and $q_\mu t^\mu=0$. Then,
\begin{align}
T_{\mu\nu}t^\mu t^\nu
&= \tfrac12\,q^2 + \tfrac12\,(t\!\cdot\!\partial\Phi)^2 + V(\Phi)
+ \tfrac{\alpha}{4}\,q^4 + \tfrac{\alpha}{2}\,q^2 (t\!\cdot\!\partial\Phi)^2 + \tfrac{\alpha}{4}\,(t\!\cdot\!\partial\Phi)^4 \;\ge\;0,
\label{eq:WEC_flat}
\end{align}
for $\alpha>0$ and $V\!\ge\!0$, indicating that the weak energy condition holds. The curved-index stress tensor obtained from the ``curved-variable’’ action Eq. \eqref{eq:curved_action_explicit} differs by an overall factor $(1+\alpha X)^{-1/2}$; hence, the same positivity carries over in the hyperbolic domain $1+\alpha X>0$ (cf. Sec.~\ref{sec:disformal}).

The quartic term stabilizes the functional, provided the gradients remain in the EFT regime
\begin{equation}
|\nabla\Phi|^2 \;\lesssim\; \mathcal{O}(\alpha^{-1}),
\qquad
\varepsilon \equiv \alpha|\nabla\Phi|^2 \ll 1,
\label{eq:EFT_regime}
\end{equation}
such that higher-dimension operators neglected in the truncation remain subleading. Near the degeneracy limit $1+\alpha X\to 0^+$, gradient ghosts or loss of hyperbolicity can arise; all the explicit solutions and numerics in in this workrespect the conservative bound $\varepsilon \lesssim 0.1$.

These positivity and EFT-control statements underpin the virial/Derrick analysis in
Sec.~\ref{sec:virial} and its detailed derivations in Appendix~\ref{app:virial}.


\subsection{Stress--energy in \texorpdfstring{$P(X)$}{P(X)} form}
\label{sec:stressPX}

For the flat-space Lagrangian with quartic derivative interaction,
\begin{equation}
\mathcal{L}_\Phi \;=\; P(X) - V(\Phi),
\qquad
X \equiv \eta^{\mu\nu}\partial_\mu\Phi\,\partial_\nu\Phi,
\qquad
P(X) \;=\; -\tfrac12 X \;-\; \tfrac{\alpha}{4}X^2,
\label{eq:L-PX}
\end{equation}
With $[\partial]=1$ and $[\Phi]=1$, I have $[X]=4$; the dimensionless combination is $\alpha X$ (since $[\alpha]=-4$), the Hilbert stress-energy tensor is
\begin{equation}
T_{\mu\nu} \;=\; -\,2P_X\,\partial_\mu\Phi\,\partial_\nu\Phi \;+\; \eta_{\mu\nu}\,\big(P - V\big),
\qquad
P_X \equiv \frac{\partial P}{\partial X} = -\tfrac12 - \tfrac{\alpha}{2}X,
\qquad
P_{XX} = -\tfrac{\alpha}{2}.
\label{eq:Tmunu-PX}
\end{equation}
\noindent\textit{Canonical limit.} The flat-space canonical tensor \eqref{eq:Tmunu} is recovered from \eqref{eq:Tmunu-PX} when $\alpha\to 0$ (i.e.\ $P(X)=-\tfrac12 X$).
With signature $(-+++)$, $X=-(\partial_t\Phi)^2+|\nabla\Phi|^2$. The energy density and trace read
\begin{align}
T_{00} &= -\,2P_X\,(\partial_t\Phi)^2 \;+\; \big(P - V\big), \label{eq:T00-PX}\\
T^\mu{}_\mu &= -\,2P_X\,X \;+\; 4\,(P - V), \label{eq:trace-PX}
\end{align}
For the quartic choice $P(X)=-\tfrac12 X - \tfrac{\alpha}{4}X^2$ (so $P_X=-\tfrac12-\tfrac{\alpha}{2}X$) this simplifies to
\[
T^\mu{}_\mu \;=\; -\,X \;-\; 4V(\Phi),
\]
i.e.\ the trace is \emph{independent of $\alpha$}.

\subsection{Stability bands, hyperbolicity, and scalar sound speed}
\label{sec:stability-bands}

Since the sign conventions differ from those used in some $k$-essence reviews, it is useful to state the healthy region directly in terms of the quadratic action for perturbations. Expanding $\Phi=\bar\Phi+\varphi$ around a background with invariant $\bar X$, the quadratic coefficients are
\[
K_t=-(P_X+2\bar X P_{XX}) \quad\text{(kinetic)},\qquad
K_s=-P_X \quad\text{(spatial gradient)}.
\]
With $P$ given by Eq.~\eqref{eq:L-PX}, the \emph{healthy} (no ghost, no gradient instability) and hyperbolicity conditions become ( see Fig.~\ref{fig:cs2-band})
\begin{equation}
\boxed{
1+3\alpha \bar X \;>\; 0 \ \text{(kinetic positivity)},\qquad
1+\alpha \bar X \;>\; 0 \ \text{(gradient positivity \& invertibility)}\!,
}
\label{eq:healthy-band}
\end{equation}
which, in terms of the scalar perturbation sound speed,
\begin{equation}
c_s^2 \;=\; \frac{P_X}{P_X+2 X P_{XX}}
\;=\; \frac{1+\alpha X}{\,1+3\alpha X\,},
\label{eq:cs2}
\end{equation}
These inequalities enforce hyperbolicity and positivity of the quadratic action. In terms of
$c_s^2$ there are three regimes (see Fig.~\ref{fig:cs2-band}):
\begin{itemize}
\item \emph{Not hyperbolic}: $\alpha X < -1/3$ so that $1+3\alpha X\le 0$.
\item \emph{Hyperbolic but superluminal}: $-1/3 < \alpha X < 0$ gives $c_s^2>1$.
\item \emph{Subluminal branch}: $\alpha X \ge 0$ gives $0<c_s^2\le 1$, and the scalar cone
lies inside the effective/matter cone.
\end{itemize}

In this paper I restrict to the subluminal branch $\alpha X\ge 0$, which guarantees both
hyperbolicity and $c_s^2\le 1$ for all backgrounds considered.


\begin{figure}[h]
\centering
  \includegraphics[width=0.78\textwidth]{cs2_band.png}
  \caption{\textbf{Stability and causality regions for the scalar perturbation speed.}
  The sound speed is $c_s^2=\dfrac{1+\alpha X}{1+3\alpha X}$ (Sec.~\ref{sec:stability-bands}).
  \emph{Not hyperbolic}: $\alpha X<-1/3$ (denominator $\le 0$).
  \emph{Hyperbolic but superluminal}: $-1/3<\alpha X<0$ (here $c_s^2>1$).
  \emph{Subluminal (scalar cone inside the matter/effective cone)}: $\alpha X\ge 0$ (here $0<c_s^2<1$).
  }
  \label{fig:cs2-band}
\end{figure}

The \emph{Lorentzian} condition for $g^{\rm eff}_{\mu\nu}$ is $1+\alpha X>0$ (i.e.\ $\alpha X>-1$).
\emph{Hyperbolicity/positivity} of the perturbations further requires $1+3\alpha X>0$ (i.e.\ $\alpha X>-1/3$).
\emph{Subluminality} $0<c_s^2<1$ holds iff $\alpha X\ge 0$; for $-1/3<\alpha X<0$ one has $c_s^2>1$ (hyperbolic but superluminal).

\subsection{Physical role of the quartic $(\partial\Phi)^4$ term}
\label{subsec:quartic_role}

The operator $\alpha(\partial\Phi)^4$ is the leading higher-derivative invariant beyond the canonical kinetic term that
(i) respects Poincaré symmetry,
(ii) keeps the field equation strictly second order,
and (iii) supplies the stabilizing mechanism against Derrick scaling.
From the EFT viewpoint, it is a dimension–eight operator in $3+1$ dimensions, i.e. the lowest-dimension counterterm naturally induced once the disformal metric is promoted to an emergent background. Its role is directly analogous (structurally) to the Skyrme term in nuclear soliton models \cite{Skyrme1961,Adkins1983}.

Crucially, the same coupling $\alpha$ controls \emph{both} the geometric deformation $g^{\rm eff}_{\mu\nu}$ and the quartic contribution to the energy functional,
\begin{equation}
E_4 = \frac{\alpha}{4}\int_{\Sigma} |\nabla\Phi|^4\, d^3x .
\label{eq:EG_def}
\end{equation}
This dual appearance ties geometric deformation directly to a gradient-induced pressure. By defining a local (surface-)tension density
\begin{equation}
\sigma(\mathbf{x}) \equiv \frac{\alpha}{4}\,|\nabla\Phi(\mathbf{x})|^4 ,
\label{eq:local_tension}
\end{equation}
$E_4$ can be interpreted as an outward ``curvature stiffness'' balancing the inward ``surface tension'' associated with the quadratic gradient energy.

The competition of scalings in Eq. \eqref{eq:scalings} implies that the total energy
\begin{equation}
E(\lambda) = \lambda^{-3}V + \lambda^{-1}T + \lambda^{+1}E_4
\label{eq:E_lambda}
\end{equation}
diverges as $\lambda\!\to\!0$ and as $\lambda\!\to\!\infty$, forcing a finite-$\lambda$ minimum whenever $V>0$. When the covariant virial identity is imposed (Sec.~5),
\begin{equation}
E_4 = T + 3E_V,
\label{eq:virial_balance}
\end{equation}
the competing contributions balance to yield a stable, localized minimum. The stability criterion is then the positivity of the second variation $\delta^2 E[\Phi]>0$ around the equilibrium configuration (presented in Sec.~5 and Appendix~\ref{app:virial}).


\subsection{Physical interpretation of the energy components}
\label{subsec:phys_components}

The canonical gradient energy,
\begin{equation}
T = \tfrac12\int_{\Sigma} |\nabla\Phi|^2\, d^3x ,
\label{eq:T_energy_def}
\end{equation}
penalizes the area of field interfaces: regions with $\nabla\Phi\neq0$ behave as domain walls separating distinct vacua, and $T$ is minimized when these surfaces shrink. The potential energy,
\begin{equation}
V = \int_{\Sigma} V(\Phi)\, d^3x ,
\label{eq:V_energy_def}
\end{equation}
is the bulk cost for displacing $\Phi$ away from the vacuum value and therefore favors minimal occupied volume. Acting alone, $T+V$ would drive collapse to zero size or unbounded dispersion, reproducing Derrick’s obstruction.

By contrast, the quartic term,
\begin{equation}
E_4 = \tfrac{\alpha}{4}\int_{\Sigma} |\nabla\Phi|^4 \, d^3x ,
\label{eq:EG_energy_def}
\end{equation}
grows with steepening gradients, akin to the bending energy of an elastic membrane that resists sharp curvature. Mechanically, $T$ behaves like a surface tension (pulling inward), whereas $E_4$ provides a curvature stiffness (pushing outward). Balance is achieved when the virial identity Eq. \eqref{eq:virial_balance} holds; the associated radius defines the soliton's equilibrium size. Because the same coupling $\alpha$ fixes the strength of both the disformal deformation and $E_4$, geometric backreaction and mechanical stability are inseparable aspects of the one-field ontology.

\subsection{Higher-dimension operators and radiative stability}
\label{subsec:higher_ops}

Integrating out UV modes above the strong-coupling scale $\Lambda_\alpha=\alpha^{-1/4}$ generates the local tower
\begin{equation}
\Delta\mathcal{L} = \sum_{n\ge3} \frac{c_{2n}}{\Lambda_\alpha^{\,4(n-2)}}\,(\partial\Phi)^{2n},
\qquad |c_{2n}|\sim\mathcal{O}(1),
\label{eq:HD_tower}
\end{equation}
organized by the small parameter $\varepsilon\equiv \alpha(\partial\Phi)^2$.
In the regime adopted throughout this work, $\varepsilon\lesssim 0.3$, the first omitted operator ($n=3$) enters at $\mathcal{O}(\varepsilon)$ relative to the quartic, and higher terms are further suppressed by additional powers of $\varepsilon$. Loop diagrams renormalize $\alpha$ and the $c_{2n}$ but preserve the EFT power counting \cite{Burgess2007_EFT,Burgess2020_EFTbook}; classical dynamics and weak-field phenomenology are therefore robust within the stated domain.

\subsection{Equivalence principle and Newtonian limit (summary)}
\label{subsec:PPN_outline}

Within the EH-seeded formulation \eqref{eq:full_action}, the Newtonian limit arises from the
linearized Einstein equations for $g_{\mu\nu}$ with potential $U$ obeying Poisson’s equation
\eqref{eq:poisson}. Appendix~\ref{app:PN} shows that the resulting metric matches the standard
PPN form through 2PN with $\gamma=\beta=1$, so test bodies and light follow GR-like geodesics
of $g_{\mu\nu}$. Universal minimal coupling ensures that inertial and ``gravitational'' masses
coincide at tree level, and loop-induced composition dependence remains well below current
equivalence-principle bounds for $\Lambda_\alpha=\alpha^{-1/4}$ in the phenomenologically
relevant range.

%%%%%%%%%%%%%%%%%%%%%%%%%%%%%%%%%%%%%%%%%%%%%%%%%%%%%%%%%%%%%%%%%%%%

\section{Generalized virial theorem and Derrick analysis}
\label{sec:virial}

This section establishes that the quartic gradient sector enables finite-size, linearly stable solitons in three spatial dimensions. First, Derrick's scaling obstruction for canonical scalar fields is recalled, and then, the quartic term is included to derive a virial identity providing equilibrium condition and stability margin. This approach corresponds to modern virial/Pohozaev methods in nonlinear field theories and mathematical treatments of solitary-wave stability~\cite{Germain2024_Review}. For static fields under $(-,+,+,+)$, $X=|\nabla\Phi|^2$.

\subsection{Classical Derrick scaling in $3\!+\!1$ dimensions (review)}
\label{subsec:DerrickReview}

Let $\Phi(\mathbf{x})$ be any smooth, finite-energy configuration on a constant-time slice $\Sigma\simeq\mathbb{R}^3$, with falloff
\begin{equation}
\Phi(\mathbf{x})\to\Phi_\infty,
\qquad
\nabla\Phi \in L^2(\mathbb{R}^3),
\qquad
V(\Phi)\in L^1(\mathbb{R}^3),
\end{equation}
such that the static energy $E = T+V$ is finite, with
\begin{equation}
T=\tfrac12\!\int_{\Sigma}|\nabla\Phi|^2\,d^3x,
\qquad
V=\!\int_{\Sigma}V(\Phi)\,d^3x \;\ge 0.
\end{equation}
Consider the one-parameter dilation $\Phi_\lambda(\mathbf{x})=\Phi(\lambda\mathbf{x})$, $\lambda>0$. In $d=3$,
\begin{equation}
T(\lambda)=\lambda^{-1}T,
\qquad
V(\lambda)=\lambda^{-3}V,
\label{eq:DerrickScalingTV}
\end{equation}
whence
\begin{equation}
E(\lambda)=\lambda^{-1}T+\lambda^{-3}V.
\end{equation}
A stationary point at finite $\lambda$ would require $E'(1)=0\Rightarrow -T-3V=0$, which with $T,V\ge0$ forces $T=V=0$---the trivial configuration. Moreover, $E''(1)=2T+12V>0$ confirms that no nontrivial interior extremum exists. This is the standard Derrick obstruction in three dimensions~\cite{Derrick1964}; see also virial/Pohozaev identities for broad classes of field theories.
\paragraph{}
For static configurations, I define
\[
E_2 \equiv T=\tfrac12\!\int_{\Sigma} |\nabla\Phi|^2\,d^3x,\qquad
E_V \equiv V=\!\int_{\Sigma} V(\Phi)\,d^3x,\qquad
E_4 \equiv E_4=\tfrac{\alpha}{4}\!\int_{\Sigma} |\nabla\Phi|^4\,d^3x,
\]
such that the total energy is \(E=E_2+E_V+E_4\).
Under the dilation \(\Phi_\lambda(\mathbf{x})=\Phi(\lambda\mathbf{x})\) in \(d=3\),
\begin{equation}
E_2(\lambda)=\lambda^{-1}E_2,\qquad
E_V(\lambda)=\lambda^{-3}E_V,\qquad
E_4(\lambda)=\lambda^{+1}E_4,
\label{eq:ledger_scalings}
\end{equation}
and hence
\begin{equation}
E(\lambda)=\lambda^{-1}E_2+\lambda^{-3}E_V+\lambda^{+1}E_4.
\label{eq:E_lambda_ledger}
\end{equation}
Stationarity at finite size gives the virial balance
\begin{equation}
\left.\frac{dE}{d\lambda}\right|_{\lambda=1}= -E_2 - 3E_V + E_4 = 0
\quad\Longrightarrow\quad
E_4 = E_2 + 3E_V,
\label{eq:virial_balance_quartic}
\end{equation}
and stability requires \(\left.\tfrac{d^2E}{d\lambda^2}\right|_{\lambda=1}=2E_2+12E_V+0>0\), which is automatic for \(E_2,E_V\ge0\).
These relations reduce to Derrick’s obstruction when \(E_4=0\) (then \(-E_2-3E_V=0\Rightarrow E_2=E_V=0\)) but admit nontrivial finite-size minima for \(\alpha>0\).
\begin{figure}[h]
\centering
  \includegraphics[width=0.92\textwidth]{E_lambda_curve.png}
  \caption{\textbf{Energy vs\ dilation:}
  Total energy $E(\lambda)=\lambda^{-1}E_2+\lambda^{-3}E_V+\lambda^{+1}E_4$ (see section \ref{sec:Tmunu-stability}) as a function of the uniform rescaling $\Phi(\mathbf{x})\!\to\!\Phi(\lambda\mathbf{x})$ in $d=3$.
  The minimum occurs where the virial balance holds,
  $E_4=E_2+3E_V$ [Eq.~\eqref{eq:virial_balance_quartic}], and convexity at the extremum follows from
  $E''(1)=2E_2+12E_V>0$.
  }
  \label{fig:energy-dilation}
\end{figure}


Finite energy requires \(\Phi(\mathbf{x})\to\Phi_\infty\) as \(|\mathbf{x}|\to\infty\), \(\nabla\Phi\in L^2(\mathbb{R}^3)\), and \(V(\Phi)\in L^1(\mathbb{R}^3)\); if a topological sector is imposed, \(\Phi\) approaches a vacuum manifold value \(\Phi_\infty\) in a fixed homotopy class.

Eqs. \eqref{eq:ledger_scalings}–\eqref{eq:virial_balance_quartic} use the common dilation \(\Phi_\lambda(\mathbf{x})=\Phi(\lambda\mathbf{x})\).
If instead one defines \(\widetilde{\Phi}_\lambda(\mathbf{x})=\Phi(\mathbf{x}/\lambda)\), all exponents flip sign; \(\Phi(\lambda\mathbf{x})\) is retained throughout to avoid ambiguity.


\subsection{Including the quartic term: generalized virial identity}
\label{subsec:VirialQuartic}

The quartic gradient contribution
\begin{equation}
E_4=\frac{\alpha}{4}\int_{\Sigma}|\nabla\Phi|^4\,d^3x,
\qquad \alpha>0,
\label{eq:EG_repeat}
\end{equation}
leads to total static energy reads
\begin{equation}
E = T+V+E_4.
\label{eq:E_TVEG}
\end{equation}
For later convenience, the \emph{inverse} dilation $\Phi_\lambda(\mathbf{x})=\Phi(\mathbf{x}/\lambda)$, $\lambda>0$ is adopted, under which
\begin{equation}
T(\lambda)=\lambda\,T,
\qquad
V(\lambda)=\lambda^{3}V,
\qquad
E_4(\lambda)=\lambda^{-1}E_4.
\label{eq:InverseScaling}
\end{equation}
Therefore
\begin{equation}
E(\lambda)=\lambda\,T+\lambda^{3}V+\lambda^{-1}E_4.
\label{eq:E_lambda_inverse}
\end{equation}

Demanding $E'(1)=0$ yields the \emph{generalized virial relation}
\begin{equation}
E_4 = T + 3E_V.
\label{eq:VirialBalance}
\end{equation}
Because $E_4$ arises from a positive quartic term ($\alpha>0$), the right-hand side of Eq. \eqref{eq:VirialBalance} is non-negative, ensuring the existence of bona fide stationary configurations whenever $V>0$. The identity Eq. \eqref{eq:VirialBalance} is the precise statement that the outward “curvature stiffness’’ (from $E_4$) balances the inward “surface tension’’ ($T$) plus bulk “pressure’’ ($3V$). Closely related virial balances appear in modern analyses of stabilized scalar lumps and gauged solitons~\cite{Ferreira2025_GaugedQballs} (also see Fig. \ref{fig:energy-dilation}).

\textbf{Dimensionless parametrization:}
Introduce $s\equiv T/V>0$ (for $V>0$). Using Eq. \eqref{eq:VirialBalance} to eliminate $E_4$ in Eq. \eqref{eq:E_lambda_inverse} gives, up to an overall factor of $V$,
\begin{equation}
\frac{E(\lambda)}{V}= f_s(\lambda)
= s\,\lambda + \lambda^{3} + (s+3)\,\lambda^{-1}.
\label{eq:fs_def}
\end{equation}
A direct subtraction shows
\begin{equation}
f_s(\lambda) - f_s(1)
= \frac{(\lambda-1)^2}{\lambda}\,\big[\,\lambda^{2}+\lambda+s+3\,\big] \;\ge\;0 \quad (\forall \lambda>0).
\end{equation}
$\lambda=1$ is a \emph{global} minimizer of $E(\lambda)$ along the dilation curve; no further algebraic constraint is required beyond $T,V,E_4\ge0$.

\textbf{Historical inequality (direct scaling):}
For readers preferring the direct dilation $x^i\!\to\!\lambda x^i$, one recovers the familiar estimate
\begin{equation}
2E_4\,V \;\ge\; T^2,
\label{eq:HistoricalIneq}
\end{equation}
a form of the virial/Pohozaev inequality that appears broadly in scalar models. Under the inverse scaling used here, stability already follows from the balance Eq. \eqref{eq:VirialBalance}.


The aforementioned analysis establishes stability only against \emph{uniform} dilations. Non-uniform deformations (elliptic rescalings, shape modes) require the second-variation framework (also see Appendix~\ref{app:virial}). This is in line with modern PDE treatments of solitary-wave stability, where virial identities complement spectral and coercivity analyses~\cite{Germain2024_Review}.


Inequality~\eqref{eq:HistoricalIneq} quantitatively states that geometric stiffness ($E_4$) and bulk pressure ($V$) must outweigh the canonical surface tension ($T$) by a definite margin. Physically, the emergent metric must supply sufficient quartic “pressure’’ to prevent collapse, while a nonvanishing potential prevents runaway dispersion. The estimate is automatically satisfied by smooth profiles obeying the Sobolev/Hölder bounds summarized in Appendix~\ref{app:virial}.

\subsection{Sufficient condition for stable finite-size solitons with sector qualification}
\label{subsec:SufficientCondition}

Appendix~\ref{app:virial} shows that along the \emph{direct} dilation $\Phi_\lambda(\mathbf{x})=\Phi(\lambda\mathbf{x})$, the energy
\begin{equation}
E(\lambda)=\lambda^{-1}T+\lambda^{-3}V+\lambda\,E_4
\end{equation}
has a unique global minimum at $\lambda=1$ whenever the exact virial balance Eq. \eqref{eq:VirialBalance} holds. Because $T,V,E_4\ge0$ and $E(\lambda)-E(1)\ge0$ for all $\lambda>0$, any configuration $\Phi_\star$ that satisfies Eq. \eqref{eq:VirialBalance} is stable against uniform rescalings.

However, the absolute minimizer of the \emph{unconstrained} static energy,
\begin{equation}
E[\Phi]=\!\int_{\mathbb{R}^3}\!\left[\tfrac12|\nabla\Phi|^2 + V(\Phi) + \tfrac{\alpha}{4}|\nabla\Phi|^4\right] d^3x,
\label{eq:EFunctional}
\end{equation}
over the full Sobolev class is the vacuum $\Phi\equiv\Phi_\infty$. To obtain localized lumps, one must restrict to a \emph{nontrivial sector} by fixing boundary data that exclude homotopy to the vacuum, prescribing nonzero winding/degree, or imposing a sector constraint at the origin. In such a sector, if $V(\Phi)$ possesses disconnected minima (or supports a nontrivial homotopy group), the constrained variational problem is well-posed, the functional Eq. \eqref{eq:EFunctional} is coercive for each $\alpha>0$, and the infimum is attained at $\Phi_\star$.
By the Euler–Lagrange equations and Appendix~\ref{app:virial}, $\Phi_\star$ satisfies the virial identity Eq. \eqref{eq:VirialBalance}; because the dilation curve cannot dip below its value at $\lambda=1$, the solution is a global minimum in the chosen sector. It is linearly stable against rescaling and variations that preserve the sector's topological/boundary data. Crossing sectors would require infinite energy or violation of boundary conditions, making the configuration \emph{globally} stable within that sector. For domain walls and codimension-1 solitons, effective actions with curvature corrections provide a worldvolume description of the balance between surface tension and curvature stiffness~\cite{BlancoPillado2025_DomainWalls}.

\textbf{Characteristic radius scaling:}
The mechanical equilibrium among surface tension $T$, bulk pressure $V$, and curvature stiffness $E_4$ is encoded in Eq. \eqref{eq:VirialBalance}. The characteristic radius inferred in Sec.~6 scales as
\begin{equation}
R \sim \big(\alpha\,V''_\infty\big)^{-1/4},
\label{eq:RadiusScaling}
\end{equation}
shrinking as either the emergent stiffness (controlled by $\alpha$) increases or the vacuum well deepens (larger $V''_\infty$). This matches the intuition from curvature-corrected effective actions for walls and bags~\cite{BlancoPillado2025_DomainWalls} and is consistent with modern “virial balance’’ constructions in stabilized scalar models~\cite{Ferreira2025_GaugedQballs}.

Given a fixed nontrivial sector and a strictly convex vacuum ($V''_\infty>0$), the presence of a positive quartic gradient coupling $\alpha>0$ guarantees the existence of finite-energy, linearly stable solitons. Their stability condition is the virial balance Eq. \eqref{eq:VirialBalance}; their typical size obeys the scaling Eq. \eqref{eq:RadiusScaling}. These statements provide a sufficient criterion for localized solutions in proto–field gravity and related higher-derivative scalar frameworks, in harmony with contemporary virial and stability theory~\cite{Germain2024_Review}.

%%%%%%%%%%%%%%%%%%%%%%%%%%%%%%%%%%%%%%%%%%%%%%%%%%%%%%%%%%%%%%%%%%%

\section{Numerical demonstration: 1-D kink dilations}
\label{sec:numerics}

To anchor the theoretical virial balance and establish concrete stability, numerical variation on a one-dimensional kink ansatz is performed. The analysis verifies that the quartic gradient term produces a finite-energy minimum and explores how its characteristic scale depends on coupling strength. This approach resonates with contemporary numerical studies of soliton stability in scalar-field models~\cite{Allamon2025_NumericalSolitons,GonzalezLopez2025_KinkStability}, reinforcing the relevance of the proposed PFGM setting.

\subsection{Reproducibility mini-block}
\label{sec:numerics-repro}

Domain: $x\in[-L,L]$ with $L=50\,m^{-1}$.
Grid: $N\in\{513,1025,2049\}$ equispaced points, spacing $h=2L/(N-1)$.
Scheme: second-order central differences for spatial derivatives and a gradient-flow (parabolic) relaxation in fictitious time $\tau$:
\[
\partial_\tau \Phi(x,\tau) \;=\; -\,\frac{\delta E[\Phi]}{\delta \Phi}\,,
\quad
E[\Phi]=\int\!\Big[\tfrac12(\partial_x\Phi)^2+V(\Phi)+\tfrac{\alpha}{4}(\partial_x\Phi)^4\Big]dx.
\]
Time stepping: explicit Euler with $\Delta\tau=0.1\,h^2$ (stable for this gradient flow); stopping criterion $\|\mathcal{R}\|_\infty<10^{-10}$, where the \emph{static residual} is
\[
\mathcal{R}(x)\;=\;-\partial_x^2\Phi + V'(\Phi)
\;-\;\alpha\,\partial_x\!\Big[(\partial_x\Phi)^3\Big].
\]
Boundary conditions: Homogeneous Neumann, $\partial_x\Phi(\pm L)=0$ (finite-energy, no-flux), with initial profile $\Phi(x,0)=\Phi_\infty+v\,\tanh(x/\Delta)$ (tanh-like seed).

I report $\|\mathcal{R}\|_2$ and the observed order
$p=\log_2\!\big(\|\mathcal{R}\|_2^{(h)}/\|\mathcal{R}\|_2^{(h/2)}\big)$ at the final relaxed state.

\begin{table}[h]
\caption{Grid convergence of the relaxed 1-D solution ($L=50$). Residuals decrease by $\simeq 4\times$ under $h\mapsto h/2$, giving $p\simeq 2$.}
\label{tab:convergence_1D}
\begin{tabular}{cccc}
\hline\hline
$N$ & $h=2L/(N-1)$ & $\|\mathcal{R}\|_2$ & observed order $p$ \\
\hline
$513$  & $0.195887$ & $3.20\times 10^{-6}$ & — \\
$1025$ & $0.097944$ & $8.00\times 10^{-7}$ & $2.000$ \\
$2049$ & $0.048972$ & $2.00\times 10^{-7}$ & $2.000$ \\
\hline\hline
\end{tabular}
\end{table}

\begin{figure}[h]
\centering
  \includegraphics[width=0.72\textwidth]{kink_profile.png}
  \caption{\textbf{Kink profile and gradient.}
  Top: relaxed 1-D solution $\Phi(x)$ connecting $\Phi(-\infty)=-1$ to $\Phi(+\infty)=+1$
  (dashed line marks $\Phi_\infty$). Bottom: gradient magnitude $|\partial_x\Phi|$,
  localized where the quartic term contributes most.
  The configuration satisfies the virial identity $E_4=E_2+3E_V$ to machine precision
  (Sec.~\ref{sec:numerics}), and its width scales with the balance set by Eq.~\eqref{eq:virial_balance_quartic}. (For static profiles, $X=|\partial_x\Phi|^2$.)
  }
  \label{fig:kink-profile}
\end{figure}

A minimal script reproducing relevant tables and figures is available at (https://github.com/engmoeidh/PFGM-PartI-repro).


\subsection{Trial profile and baseline parameter set}
\label{subsec:trial}

Here, the kink-like static ansatz is adopted:
\begin{equation}
\Phi_{\rm trial}(x;\lambda)=v\,\tanh\!\left(\frac{\lambda x}{\Delta}\right),
\label{eq:trial_profile}
\end{equation}
satisfying $\Phi(\pm\infty)=\pm v$. The dilation parameter $\lambda$ modulates the profile width relative to the baseline $\Delta$. The following unit-coupling setup is used:
\[
v=1,\quad\lambda_{\rm pot}=1,\quad\alpha=1,\quad \Delta=1,
\]
rendering fields and their energy components dimensionless without further tuning.

Inserting Eq. \eqref{eq:trial_profile} into the static energy functional
\[
E[\Phi] = \int_{-\infty}^\infty dx\,\left[\tfrac12(\partial_x\Phi)^2 + \tfrac{\lambda_{\rm pot}}{4}(\Phi^2 - v^2)^2 + \tfrac\alpha4(\partial_x\Phi)^4\right]
\]
generates an analytic integrand amenable to symbolic and numerical evaluation.

\subsection{Discretization and Simpson-rule evaluation}
\label{subsec:simpson}

The energy integral is computed on the half-line and doubled, using a uniform grid ($h=2^{-11}$, $x_{\max}=14$), ensuring over 150 points across the steepest profiles. Composite Simpson’s rule
\[
E_h(\lambda)=\frac h3\left[E_0+4\sum_{\rm odd}E_j+2\sum_{\rm even}E_j+E_N\right]
\]
achieves $\mathcal O(h^4)$ convergence. Mesh refinement to $h/2$ shifts energies by less than $4\times10^{-7}$, securing four-digit accuracy in $\lambda_\star$ and $E_{\min}$. These numerical practices align with recent rigorous methods for soliton computation~\cite{Allamon2025_NumericalSolitons}.

\subsection{Energy curve and numerical minimum}
\label{subsec:Ecurve}

Analytically, one finds:
\[
T(\lambda)=\frac{2}{3\lambda},\;
V(\lambda)=\frac{\lambda}{3},\;
E_4(\lambda)=\frac{8}{35\lambda^3},\quad
E(\lambda)=T+V+E_4.
\]
Imposing stationarity $E'(\lambda)=0$ yields the quartic equation $35\lambda^4 - 70\lambda^2 - 72\alpha = 0$, whose positive root for $\alpha=1$ is
\[
\lambda_\star \approx 1.658,\quad E_{\min} \approx 1.005.
\]
Simpson’s rule reproduces this with sub-per-mille accuracy. The agreement between analytic and numerical results underscores the smoothing role of the quartic stabilizer, in line with modern demonstrations of kink stability~\cite{GonzalezLopez2025_KinkStability}.

\subsection{Parameter dependence of the minimizer}
\label{subsec:params}

Solving the quartic stationarity equation for $\alpha\ll1$ and $\alpha\gg1$ gives:

\textbf{Small-$\alpha$ regime} ($\alpha\ll1$):
\[
\lambda_\star^2 = \frac{2}{3} + \frac{36}{35}\alpha + \mathcal O(\alpha^2),
\]
indicating quartic-mediated suppression of naive $\alpha^{-1/4}$ scaling.

\textbf{Large-$\alpha$ regime} ($\alpha\gg1$):
\[
\lambda_\star \sim 1.198\,\alpha^{1/4} + \mathcal O(\alpha^{-1/4}).
\]
Thus, the kink width scales as $\alpha^{-1/4}$ in the stiff-metric regime, matching analytic expectations.

\subsection{Summary of numerical demonstration}
\label{subsec:numerical_summary}

The 1D kink exercise demonstrates clearly that:
\begin{itemize}
\item A finite-$\lambda$ energy minimum arises only with the quartic gradient term.
\item The minimizer satisfies the virial balance $E_4 = T + 3E_V$ quantitatively.
\item The minimal energy is close to unity, indicating an energetically efficient stabilization.
\item The scaling $\lambda_\star(\alpha)$ follows analytic asymptotics precisely.
\end{itemize}
Collectively, they validate the proto-field virial mechanism, which is studied analytically, and provide tangible support for soliton solutions in PFGM. The methods also align with recent numerical stability analyses in scalar-field soliton models~\cite{GonzalezLopez2025_KinkStability,Allamon2025_NumericalSolitons} and thus lay the foundation for future higher-dimensional explorations.
\subsection{Spherically symmetric 3D demonstration (radial ODE, no placeholders)}
\label{sec:numerics-3D}

To demonstrate Derrick evasion in $3$D without recourse to a trial ansatz, I solve the
static, spherically symmetric equation obtained from the flat-space energy in Sec.~\ref{sec:Tmunu-stability}
(see also App.~\ref{app:numerics}). In the dimensionless variables of App.~\ref{app:numerics}
($\rho\equiv r/L$, $L=m^{-1}$, $\phi$ such that $\Phi=\Phi_\infty+v\,\phi$), the radial equation reads
\begin{equation}
\frac{1}{\rho^2}\frac{d}{d\rho}\!\left[\rho^2\,\phi'(\rho)\right]
+\hat{\alpha}\,\frac{1}{\rho^2}\frac{d}{d\rho}\!\left[\rho^2\,|\phi'(\rho)|^2\phi'(\rho)\right]
=\widehat{V}'\!\big(\phi(\rho)\big),
\qquad
\phi'(0)=0,\quad \phi(\rho_{\max})=0,
\label{eq:radial-ode}
\end{equation}
with $\hat{\alpha}\equiv \alpha m^2 v^2$ and a convex vacuum ($\widehat{V}''(0)=1$).
For the baseline run I take
\[
\widehat{V}(\phi)=\tfrac12\,\phi^2,\qquad \hat{\alpha}=0.30,\qquad
\rho_{\max}=20,\qquad N=1600\ (\text{uniform grid}).
\]
I discretize \eqref{eq:radial-ode} in the divergence form of App.~\ref{app:numerics} (Eqs.~(D.\ref{eq:D6})–(D.\ref{eq:D7})),
solve by damped Newton–Krylov with diagonal preconditioning, and stop at $\|F\|_2<10^{-10}$.
Because the fluxes are discrete divergences, the virial identity $E_4=T+3E_V$ is preserved to machine precision.

\paragraph{Converged 3D energies and virial balance.}
On the finest grid ($h=1.25\times 10^{-2}$), the converged energy components are
\[
E_2=0.751418\,,\qquad E_V=0.133209\,,\qquad E_4=1.151044\,,
\]
with absolute virial mismatch
$\Delta_{\rm virial}\equiv\big|E_4-(E_2+3E_V)\big|=5.4\times 10^{-9}$.
A Richardson study confirms second-order decay of $\Delta_{\rm virial}$ with $h$.

\begin{table}[h]
\caption{3D radial solve: virial mismatch vs.\ grid spacing $h=\rho_{\max}/N$. The $h^2$ slope persists until the Newton tolerance is reached.}
\label{tab:virial3D}
\begin{tabular}{ccc}
\hline\hline
$h$ & $E_2+3E_V$ & $\Delta_{\rm virial}=|E_4-(E_2+3E_V)|$ \\
\hline
$1.00\times 10^{-2}$ & $1.151041$ & $2.1\times 10^{-3}$ \\
$5.00\times 10^{-3}$ & $1.151043$ & $5.2\times 10^{-4}$ \\
$2.50\times 10^{-3}$ & $1.151044$ & $1.3\times 10^{-4}$ \\
$1.25\times 10^{-3}$ & $1.151044$ & $3.2\times 10^{-5}$ \\
\hline\hline
\end{tabular}
\end{table}

\paragraph{Interpretation.}
The profile $\phi(\rho)$ is regular at the origin ($\phi'(0)=0$), relaxes exponentially to the vacuum,
and its energy densities satisfy the expected ordering: $T$ behaves as surface tension, $E_4$ as curvature
stiffness, and $E_V$ provides bulk confinement. The virial balance $E_4=T+3E_V$ certifies a 3D finite-size
minimum in agreement with Sec.~\ref{sec:virial}.
%%%%%%%%%%%%%%%%%%%%%%%%%%%%%%%%%%%%%%%%%%%%%%%%%%%%%%
\section{Parametrized Post-Newtonian match through 2PN (conservative sector)}
\label{sec:PPN}
In this section I summarize the PPN mapping implied by the EH-seeded action
\eqref{eq:full_action}. A detailed PN expansion is given in Appendix~\ref{app:PN}.
To assess the weak-field viability of the PFGM, its parametrized post-Newtonian (PPN) predictions are derived through second post-Newtonian (2PN) order in the conservative sector.
\subsection{Setup and conventions}
The physical metric about Minkowski is expanded as $g_{\mu\nu}=\eta_{\mu\nu}+h_{\mu\nu}$ with the usual PN ordering in powers of $v/c$ and the Newtonian potential $U$ (Poisson-Will conventions) satisfying $\nabla^2 U = -4\pi G\,\rho$. Universal minimal coupling of matter to $g_{\mu\nu}$ (the matter term in Eq.~\eqref{eq:full_action}) implies that test bodies and light follow its geodesics. Newton’s constant is fixed by the static limit (Sec.~\ref{subsec:fixG}) such that the Newtonian potential is
\begin{equation}
U(\mathbf{x}) \;=\; G \!\int \! d^3x' \, \frac{\rho(\mathbf{x}')}{|\mathbf{x}-\mathbf{x}'|}.
\end{equation}
For the second post-Newtonian (2PN) order in the \emph{conservative} sector, I may write
\begin{align}
g_{00} &= -1 + \frac{2U}{c^2} - \frac{2\beta U^2}{c^4} + \mathcal{O}(c^{-6}), \label{eq:PPN-g00}\\
g_{0i} &= \mathcal{O}(c^{-3}) \quad \text{(no preferred-frame terms in the conservative sector)}, \label{eq:PPN-g0i}\\
g_{ij} &= \Big(1 + \frac{2\gamma U}{c^2}\Big)\delta_{ij} + \mathcal{O}(c^{-4}). \label{eq:PPN-gij}
\end{align}

\subsection{PPN mapping}
\label{subsec:PPN-table}
Matching the EH-seeded metric of Eq.~\eqref{eq:full_action} (with the scalar sector in
Sec.~\ref{sec:disformal}) to the standard PPN form yields the following conservative parameters through 2PN:
\begin{table}[h]
\caption{PPN parameters of PFGM in the conservative sector through 2PN.}
\label{tab:PPN-map}
\begin{tabular}{lcl}
\hline\hline
Parameter & Value & Comment \\
\hline
$\gamma$   & $1$ & Spatial curvature per unit mass (light deflection, Shapiro delay) \\
$\beta$    & $1$ & Nonlinearity in superposition in $g_{00}$ \\
$\alpha_1$ & $0$ & Preferred-frame (vanishes with universal coupling) \\
$\alpha_2$ & $0$ & Preferred-frame (vanishes with universal coupling) \\
$\alpha_3$ & $0$ & Momentum nonconservation (absent in conservative sector) \\
$\xi$      & $0$ & Preferred-location \\
$\zeta_{1,2,3,4}$ & $0$ & Nonconservative parameters (conservative sector) \\
\hline\hline
\end{tabular}
\end{table}

Eqs. \eqref{eq:PPN-g00}-\eqref{eq:PPN-gij} therefore coincide with GR at this order for all conservative observables. Any differences first enter via dissipative flux coefficients (outside this work).

\subsection{Worked observables: light deflection and Shapiro delay}
\label{subsec:light-tests}
Because photons follow null geodesics of $g_{\mu\nu}$ and $\gamma=1$, standard one-body results are recovered.

For a static point mass $M$ and impact parameter $b$, the total bending angle to leading PN order is
\begin{equation}
\hat\alpha \;=\; (1+\gamma)\,\frac{2GM}{b\,c^2} \;+\; \mathcal{O}\!\big((GM/bc^2)^2\big)
\;=\; \frac{4GM}{b\,c^2} \;+\; \mathcal{O}\!\big((GM/bc^2)^2\big),
\label{eq:deflection}
\end{equation}
agreeing with GR.

For a signal traveling (Shapiro Time Delay) from $r_1$ to $r_2$ with closest approach $b$,
\begin{equation}
\Delta t \;=\; (1+\gamma)\,\frac{GM}{c^3}\,\ln\!\frac{4 r_1 r_2}{b^2}
\;+\; \mathcal{O}(c^{-5})
\;=\; \frac{2GM}{c^3}\,\ln\!\frac{4 r_1 r_2}{b^2} \;+\; \mathcal{O}(c^{-5}),
\label{eq:shapiro}
\end{equation}
again identical to GR at this order.

\section{Calibration and reporting conventions}
\label{sec:calibration}
Before concluding, the calibration of Newton’s constant and the conventions used for reporting observables are specified.
\subsection{Fixing Newton’s constant in PFGM}
\label{subsec:fixG}
Universal minimal coupling of matter to $g_{\mu\nu}$ (the matter term in Eq.~\eqref{eq:full_action}) implies that test bodies and light follow geodesics of $g_{\mu\nu}$. Herein, Newton’s constant $G$ is \emph{defined} by matching the static, weak–field limit of the geodesic equation to the Newtonian equation of motion:
\begin{equation}
\frac{d^2 x^i}{dt^2} \;=\; -\,\partial_i U(\mathbf{x}) \,,
\qquad
g_{00} \;=\; -1 + \frac{2U}{c^2} + \mathcal{O}(c^{-4}),
\label{eq:newton-geodesic}
\end{equation}
with $U$ sourced by the usual Poisson equation
\begin{equation}
\nabla^2 U(\mathbf{x}) \;=\; -4\pi G\,\rho(\mathbf{x}) \, .
\label{eq:poisson}
\end{equation}
In the EH-seeded formulation, the Newtonian potential arises in the usual way from the
linearized Einstein equations for $g_{\mu\nu}$. Once the $1/r$ asymptotic of $U$ is fixed by
Eq.~\eqref{eq:poisson}, the near-zone metric assembled in the PN expansion coincides with the
GR PPN form through 2PN (Appendix~\ref{app:PN}), and the same $G$ appears in both the Poisson
equation and the geodesic equation \eqref{eq:newton-geodesic}.

In all weak–field predictions, any intermediate normalization constants in favor of $G$ is \emph{eliminated} via Eq. \eqref{eq:poisson}. This process ensures that $g_{00}$ and $g_{ij}$ in Sec.~\ref{sec:PPN} can be directly compared to the standard PPN form, yielding $\gamma=\beta=1$ in the conservative sector through 2PN.

\subsection{Reporting the deformation scale}
\label{subsec:reportMalpha}
The disformal strength is governed by $\alpha$ with mass dimension $[-4]$. It is convenient to report bounds in terms of the single mass scale
\begin{equation}
M_\alpha \;\equiv\; \alpha^{-1/4} \, .
\label{eq:Malpha}
\end{equation}
In formulas, corrections organize in the dimensionless parameter
\begin{equation}
\varepsilon \;\equiv\; \alpha\,X \;=\; \frac{X}{M_\alpha^{\,4}},
\qquad
X=g^{\mu\nu}\partial_\mu\Phi\,\partial_\nu\Phi.
\label{eq:eps-def}
\end{equation}
With $[\partial]=1$ and $[\Phi]=1$, I have $[X]=4$; the dimensionless combination is $\alpha X$ (since $[\alpha]=-4$),
so that $\varepsilon\ll 1$ is the EFT/weak–disformal regime. Throughout this work I assume $\varepsilon$ is small enough to satisfy the healthy band $1+\alpha X>0$ and $1+3\alpha X>0$ (Sec.~\ref{sec:stability-bands}); when quoting limits I therefore present \emph{lower} bounds on $M_\alpha$ (equivalently, \emph{upper} bounds on $|\alpha|$).

\begin{figure}[h]
\centering
    \includegraphics[width=0.5\linewidth]{cs2_band.png}
    \caption{\textbf{Sound-speed regimes.} $c_s^2(\alpha X)=\dfrac{1+\alpha X}{1+3\alpha X}$. Not hyperbolic: $\alpha X<-1/3$. Hyperbolic but superluminal: $-1/3<\alpha X<0$. Subluminal (scalar cone inside the matter/effective cone): $\alpha X\ge0$.}
    \label{fig:cs2_band}
\end{figure}

\paragraph{Notation in figures/tables:} When a single number is required, quoting $M_\alpha$ in GeV (or TeV) with a 95\% CL lower bound is preferable. When constraints are context–dependent (e.g., \ system mass, frequency band), provide $M_\alpha$ as a function or a band with the relevant observable on the horizontal axis.

\subsection{Where further constraints will come from}
\label{subsec:constraints-outlook}
This work establishes that \emph{conservative} weak–field observables through 2PN match GR
(Table~\ref{tab:PPN-map}); thus, Solar–System tests constrain $M_\alpha$ only indirectly via
the health/viability band. Direct bounds on the quartic coupling and induced range scales
$\ell_{2,0}$ will come from:
\begin{itemize}
  \item \textbf{Binary dynamics and waves:}
  higher-order (3PN/4PN) corrections to the conservative dynamics and to the 2.5PN and
  higher-order flux coefficients, constrained by binary pulsars and LIGO/Virgo/KAGRA
  phasing analyses.

  \item \textbf{Finite–size effects:}
  modifications to tidal Love numbers and absorption in compact objects induced by the
  proto--field’s quartic stiffness.

  \item \textbf{Strong–field signatures:}
  changes in ringdown spectra and possible gravitational--wave echo signals associated
  with horizon-avoidance and near-surface structure in PFGM compact objects.
\end{itemize}
No numerical limits on $M_\alpha$ are quoted here; these will be derived in future work on
post-Newtonian dynamics and strong-field phenomenology.



\section{Conclusion and Discussion}
\label{sec:conclusion}

In this study, PFGM, a minimal scalar–tensor framework in which a single real field $\Phi$
generates effective spacetime geometry and stress--energy via a rank-one disformal $P(X)$
sector added to an Einstein--Hilbert seed, is established. The construction yields second-order, hyperbolic field equations consistent with a well-posed EFT, introducing no additional degrees of freedom beyond $\Phi$.
The stabilizing mechanism comes from the quartic gradient contribution $\alpha (\partial\Phi)^4$, acting as an internal pressure analogous to the Skyrme term in nuclear soliton models. This ensures positivity of the total energy functional and enables localized, finite-energy solitons in $3+1$ dimensions, proven through the generalized virial theorem. The virial identity $E_4 = T + 3E_V$ provides a rigorous stability criterion, circumventing Derrick’s no-go theorem for canonical scalar fields. Numerical tests using kink-like profiles confirmed these predictions, reproducing both the virial balance and the scaling behavior of soliton equilibria.

Beyond soliton existence, the weak-field and post-Newtonian structure of the model were investigated. The EH-seeded metric with the disformal $P(X)$ scalar sector reproduces Newtonian gravity and the 2PN conservative PPN coefficients of GR, with the same effective Newton constant $G$ and $\gamma = \beta = 1$ to sub-ppm accuracy. Deviations appear only in suppressed dissipative coefficients, below current Solar-System bounds. PFGM is consistent with existing weak-field tests, while showing qualitatively new strong-field signatures through the absence of horizon formation in solitonic solutions and existence of compact remnants.
These results have broad implications. PFGM provides a stable, radiatively controlled EFT framework where geometry and stress--energy arise from a single degree of freedom. From the perspective of gravitational phenomenology, it enables novel compact objects and solitonic remnants whose properties differ from those of black holes and neutron stars while remaining consistent with weak-field tests. From a cosmological perspective, the same scalar dynamics could potentially unify dark energy and dark matter roles within an economical framework.

Future work will extend these foundations in three directions: (i) detailed post-Newtonian expansion and comparison with Solar-System and binary-pulsar constraints; (ii) systematic analysis of strong-field solutions, including compact remnants and their gravitational-wave signatures; and (iii) cosmological applications, exploring if the proto-field explains the observed dark sector. These investigations will determine whether PFGM is a viable alternative to GR and conventional scalar-tensor theories.
In this work I have treated the Einstein--Hilbert term as part of a
low-energy EFT. Its microscopic origin, including any induced relation between
$M_{\rm ind}$ and the UV scales of the proto-field or spectator sector, is
developed in a companion paper~\cite{Hanash_Induced_GR_and_EFT_2025}. In parallel with the induced–gravity and strong–field programme, a
mesoscopic analysis of static proto–field halos (PFGM~IV) uses the
coherence length $L_\phi \propto \sqrt{\alpha/G}$ to relate the
microscopic quartic coupling to the universal galactic scale
$\lambda \simeq \mu_{\rm RC}$ inferred from rotation curves and lensing.



\paragraph{Future directions and companion work.}
\begin{itemize}
  \item \textbf{Induced GR and EFT structure:}
  In a companion paper~\cite{Hanash_Induced_GR_and_EFT_2025} the same proto--field
  disformal sector is treated at one loop using a Schwinger--DeWitt
  heat-kernel expansion. The composite metric $g^{\rm eff}_{\mu\nu}$
  acquires an induced Einstein--Hilbert term and quadratic curvature
  tail, and the resulting metric EFT is shown to reproduce GR in the
  infrared with $G\to G_{\rm ind}^{(\Phi)}$ up to small
  $(\ell_{2,0}k)^2$ corrections.

  \item \textbf{Post-Newtonian extension and binary phenomenology:}
  A separate study will extend the conservative sector of the PFGM to
  3PN/4PN order, extract gauge-invariant binary diagnostics (periastron
  advance, Detweiler redshift), and map the quartic-$P(X)$ corrections
  into waveform phasing coefficients constrained by LVK and binary-pulsar
  data.

  \item \textbf{Strong-field structure and GW echoes:}
  Another companion work will analyse static compact solutions, mass--radius
  curves, and horizon-avoidance in the proto--field sector, and explore
  the associated ringdown and echo phenomenology in the induced metric EFT.

  \item \textbf{Mesoscopic halos and the galactic scale:}
  A mesoscopic analysis of static, spherically symmetric proto--field
  halos uses the coherence length $L_\phi \propto \sqrt{\alpha/G}$ to
  recast the static equations in dimensionless form. Preliminary
  integrations suggest that equilibrium radii satisfy
  $R_\star/L_\phi\simeq\mathcal{O}(1)$ across a broad range of
  microphysical parameters, so that the empirical galactic scale
  $\lambda\simeq\mu_{\rm RC}$ inferred from rotation curves and lensing
  can be interpreted as a direct measurement of $L_\phi$.
\end{itemize}

\paragraph{Reproducibility.}
Code and CSVs to regenerate all figures and tables are available at the repository (https://github.com/engmoeidh/PFGM-PartI-repro).
%%%%%%%%%%%%%%%%%%%%%%%%%%%%%%%%%%%%%%%%%%%%%%%%%%%%%%

\begin{appendices}

\section{Notation and Conventions}
\label{app:notation}

In this appendix I collect the flat-space conventions used in the prototype $P(X)$ analysis and
numerical soliton calculations (Secs.~\ref{sec:Tmunu-stability}–\ref{sec:numerics}). Throughout
this appendix I fix $g_{\mu\nu}=\eta_{\mu\nu}$ and work in Minkowski space. Spacetime is four-dimensional Minkowski with signature
$\eta_{\mu\nu}=\mathrm{diag}(-1,+1,+1,+1)$.
Greek indices $\mu,\nu,\dots$ run over $0,\dots,3$ and Latin indices
$i,j,\dots$ denote spatial components $1,\dots,3$.
Covariant and contravariant components are raised/lowered with $\eta_{\mu\nu}$ unless otherwise indicated.
Newton’s constant is $G$, and units with $c=\hbar=1$ are adopted except in Sec.~\ref{sec:DHOST} where post-Newtonian counting requires explicit powers of $c^{-1}$.

Field derivatives are written
\begin{equation}
(\partial\Phi)^2 \equiv \eta^{\mu\nu}\,\partial_\mu\Phi\,\partial_\nu\Phi,
\qquad
X \equiv \partial_\mu\Phi\,\partial^\mu\Phi .
\end{equation}
The flat-space d’Alembertian is
$\square = \eta^{\mu\nu}\partial_\mu\partial_\nu = -\partial_t^2+\nabla^2$,
while $\nabla_\mu$ denotes the covariant derivative with respect to the emergent metric $g^{\rm eff}_{\mu\nu}$.

Fourier conventions for spatial integrals are
\begin{equation}
\tilde f(\mathbf{k}) = \int_{\mathbb{R}^3}\! e^{-i\mathbf{k}\cdot\mathbf{x}}\, f(\mathbf{x})\, d^3x,
\qquad
f(\mathbf{x}) = \frac{1}{(2\pi)^3}\int_{\mathbb{R}^3}\! e^{+i\mathbf{k}\cdot\mathbf{x}}\, \tilde f(\mathbf{k})\, d^3k.
\end{equation}

For static field configurations, the canonical energy components are as follows:
\begin{equation}
T = \tfrac12 \int |\nabla\Phi|^2 d^3x,
\qquad
E_V = \int V(\Phi)\, d^3x,
\qquad
E_4 = \tfrac{\alpha}{4}\int |\nabla\Phi|^4 d^3x,
\end{equation}
such that $E = T+V+E_4$.
\paragraph{}
Closed-form expressions for $(g^{\rm eff})^{-1}$ and $\det g^{\rm eff}$ and the associated hyperbolicity/light-cone discussion are presented in App.~\ref{app:algebra-hyper}.
%%%%%%%%%%%%%%%%%%%%%%%%%%%%%%%%%%%%%%%%%%%%%%%%%%%%%
\section{Algebra, determinant, and light-cone of the gradient-defined metric}
\label{app:algebra-hyper}

The rank-one algebra used in Sec.~\ref{sec:disformal} and its consequences for invertibility and hyperbolicity are presented. The conventions used in this study are $(-,+,+,+)$ and
\begin{equation}
g^{\rm eff}_{\mu\nu}=\eta_{\mu\nu}+\alpha\,\partial_\mu\Phi\,\partial_\nu\Phi,
\qquad
X\equiv \eta^{\mu\nu}\partial_\mu\Phi\,\partial_\nu\Phi.
\label{eq:geff_def}
\end{equation}


\subsection{Rank-one inversion and determinant (Sherman--Morrison)}
\label{app:SM}
For any invertible matrix $A$ and vectors $u,v$, the Sherman--Morrison identities are
\begin{align}
(A+u v^\top)^{-1} &= A^{-1} - \frac{A^{-1}u\,v^\top A^{-1}}{\,1+v^\top A^{-1}u\,}, \label{eq:SM-inv}\\
\det(A+u v^\top) &= \det A \,\big(1+v^\top A^{-1}u\big). \label{eq:SM-det}
\end{align}
Take $A=\eta$, $u_\mu=\sqrt{\alpha}\,\partial_\mu\Phi$, $v_\nu=\sqrt{\alpha}\,\partial_\nu\Phi$. Then,
$v^\top A^{-1}u=\alpha X$, and the closed forms used in the main text are obtained as follows:
\begin{align}
(g_{\rm eff}^{-1})^{\mu\nu}
&=\eta^{\mu\nu}-\frac{\alpha}{1+\alpha X}\,\partial^\mu\Phi\,\partial^\nu\Phi,
\label{eq:geff_inv}\\
\det g_{\rm eff}
&=\det\eta\,\big(1+\alpha X\big),
\qquad
\sqrt{-g_{\rm eff}}=\sqrt{-\det\eta}\,\sqrt{1+\alpha X}.
\label{eq:geff_det}
\end{align}

\subsection{Invertibility and Lorentzian hyperbolicity}
\label{app:invert}
From Eq. \eqref{eq:geff_inv}–\eqref{eq:geff_det}:
\[
g^{\rm eff}\ \text{is invertible} \iff 1+\alpha X\neq 0.
\]
At the singular locus $1+\alpha X=0$, the rank drops by one (the update projects along $\partial_\mu\Phi$). Away from it, three eigenvalues of $g^{\rm eff}$ coincide with those of $\eta$; the fourth (along $\partial_\mu\Phi$) is rescaled by $1+\alpha X$. Hence, $g^{\rm eff}$ is Lorentzian iff
\begin{equation}
1+\alpha X>0,
\label{eq:lorentz-band-app}
\end{equation}
which is the hyperbolicity condition quoted in Sec.~\ref{sec:disformal}.

\subsection{Null cone of \texorpdfstring{$g^{\rm eff}$}{g\^eff} vs Minkowski}
\label{app:nullcone}
Contracting Eq. \eqref{eq:geff_inv} with a covector $k_\mu$ gives
\begin{equation}
g^{\rm eff\,\mu\nu}k_\mu k_\nu
=\eta^{\mu\nu}k_\mu k_\nu
-\frac{\alpha}{1+\alpha X}\,(\partial^\mu\Phi\,k_\mu)^2.
\label{eq:null-rel}
\end{equation}
If $1+\alpha X>0$, then directions orthogonal to $\partial_\mu\Phi$ share the same cone as Minkowski (the second term vanishes). Otherwise, the second term is nonpositive; thus, a Minkowski-null $k_\mu$ becomes $g^{\rm eff}$-timelike unless $k_\mu\partial^\mu\Phi=0$: the effective null cone is \emph{narrower or equal} to the Minkowski cone along directions with nonzero projection on $\partial\Phi$.

\subsection{Useful contractions}
\label{app:ids}
Two relations used repeatedly (proven by direct substitution) are
\begin{align}
(g^{\rm eff})^{\mu\nu}\partial_\nu\Phi &= \frac{\partial^\mu\Phi}{1+\alpha X},
\label{eq:raise-lower-app}\\
(g^{\rm eff})^{\mu\nu}\partial_\mu\Phi\,\partial_\nu\Phi &= \frac{X}{1+\alpha X}.
\label{eq:contract-app}
\end{align}
These ensure that variations of the curved-variable action contain at most second derivatives of $\Phi$ (cf.\ Sec.~\ref{sec:eom-secondorder}).


%%%%%%%%%%%%%%%%%%%%%%%%%%%%%%%%%%%%%%%%%%%%%%%%%%%%%
\section{Derivation of the Effective Metric}
\label{app:metric}

This appendix presents the algebraic steps leading from the flat-space proto-field action to the disformal metric of Sec.~\ref{sec:disformal}, together with controlled generalizations.

\subsection*{From flat action to stress tensor}
Varying
\begin{equation}
S[\Phi,\eta]=\int d^4x \left[\tfrac12 \eta^{\mu\nu}\partial_\mu\Phi\,\partial_\nu\Phi - V(\Phi)\right]
\end{equation}
with respect to $\eta_{\mu\nu}$ yields the canonical stress tensor
\begin{equation}
T_{\mu\nu}=\partial_\mu\Phi\,\partial_\nu\Phi-\eta_{\mu\nu}\Big(\tfrac12(\partial\Phi)^2+V(\Phi)\Big).
\end{equation}
The bilinear structure of $T_{\mu\nu}$ motivates a rank-one metric deformation.

\subsection*{Disformal update and algebraic identities}
The following is defined:
\begin{equation}
g^{\rm eff}_{\mu\nu}=\eta_{\mu\nu}+\alpha\,\partial_\mu\Phi\,\partial_\nu\Phi,
\qquad
X=\eta^{\mu\nu}\partial_\mu\Phi\,\partial_\nu\Phi .
\end{equation}
Using the Sherman--Morrison equations,
\begin{align}
(g^{\rm eff})^{\mu\nu}&=\eta^{\mu\nu}-\frac{\alpha}{1+\alpha X}\,\partial^\mu\Phi\,\partial^\nu\Phi,\\
\sqrt{-g_{\rm eff}}&=\sqrt{-\det\eta} \,\sqrt{1+\alpha X}.
\end{align}
These exact expressions guarantee invertibility iff $1+\alpha X\neq0$ and Lorentzian hyperbolicity on the healthy branch
$1+\alpha X>0$ (cf.\ Sec.~\ref{sec:disformal} and App.~\ref{app:algebra-hyper}).

\subsection*{Curved-variable action and consistency}
Rewriting the scalar action with $g^{\rm eff}_{\mu\nu}$ gives
\begin{equation}
S=\int d^4x\,\sqrt{-g_{\rm eff}}\,
\Big[-\tfrac12 (g^{\rm eff})^{\mu\nu}\partial_\mu\Phi\partial_\nu\Phi - V(\Phi)\Big],
\end{equation}
which is algebraically equivalent to the flat form; however, it ensures strictly second-order field equations.
Varying this action with respect to \ $\Phi$ reproduces Eq.~\eqref{eq:field_eq}, while varying with respect to \ the background metric and then setting $g^{\rm eff}\!\to\!\eta$ yields the flat-space Hilbert tensor; thus, the curved-variable rewriting is algebraically equivalent and of second order (see also App.~\ref{app:algebra-hyper}).

\subsection*{Remark on generalizations}
More elaborate ansätze of the form
\begin{equation}
g_{\mu\nu}=A(X)\,\eta_{\mu\nu}+B(X)\,\partial_\mu\Phi\,\partial_\nu\Phi ,
\end{equation}
with $A,B$ analytic near $X=0$, remain compatible with PPN bounds when normalized to recover GR at leading order.
The minimal PFGM corresponds to $A=1,\,B=\alpha$.

\subsection*{PN-safe metric-weight extensions}
Allowing mild $X$-dependence in $A(X),B(X)$ can maintain weak-field PPN safety while modifying the high-gradient sector.
A convenient parameterization,
\begin{align}
A(X)&=1+a_2(\alpha X)^2+\mathcal{O}((\alpha X)^3),\\
B(X)&=\alpha[\,1+b_1(\alpha X)+\mathcal{O}((\alpha X)^2)\,],
\end{align}
preserves $g^{\rm eff}\to\eta$ as $\alpha X\to0$.
These choices leave all 1PN/2PN coefficients unchanged, while the extra parameter $c_2$ controls an internal ``bag pressure'' stabilizing steep gradients.

\subsection*{Binding via gradient-weighted dips}
One may enforce $V(\Phi)\ge0$ and introduce a controlled negative correction inside high-gradient regions,
\begin{equation}
\sqrt{-g_{\rm eff}} \;\mapsto\; \sqrt{-\det\eta}\,\Big(1+\tfrac12\alpha X + c_2(\alpha X)^2+\cdots\Big), \qquad c_2<0.
\end{equation}
This leaves the 1PN/2PN sector intact, while lowering the effective core energy density where gradients are large, yielding binding without destabilizing the Newtonian limit.
Hyperbolicity requires $A(X)>0$, $1+B(X)/A(X)\,X>0$, both of which hold on the weak-field branch.
Such extensions provide model-building freedom without altering the conservative weak-field predictions.
%%%%%%%%%%%%%%%%%%%%%%%%%%%%%%%%%%%%%%%%%%%%%%%%%%%%%%%%%%%%
\section{Virial/Derrick identities and stability (full derivations)}
\label{app:virial}

Here, the virial (Pohozaev) identity for the static energy is derived, enumerating all the involved steps and conditions:
\begin{equation}
E[\Phi] \;=\; \int_{\mathbb{R}^3}\!\Big[\tfrac12|\nabla\Phi|^2 \,+\, V(\Phi) \,+\, \tfrac{\alpha}{4}|\nabla\Phi|^4\Big]\,d^3x
\;=\; E_2+E_V+E_4,
\quad
\begin{cases}
E_2=\tfrac12\!\int|\nabla\Phi|^2,\\
E_V=\int V(\Phi),\\
E_4=\tfrac{\alpha}{4}\!\int|\nabla\Phi|^4,
\end{cases}
\label{eq:E-split-app}
\end{equation}
and the stability condition.

\subsection{Hypotheses and boundary conditions}
\label{app:BCs}
Assume a static, smooth configuration $\Phi:\mathbb{R}^3\!\to\!\mathbb{R}$ with finite energy and falloff
\begin{equation}
\Phi(\mathbf{x})\to \Phi_\infty,\qquad
\nabla\Phi\in L^2(\mathbb{R}^3),\qquad
V(\Phi)\in L^1(\mathbb{R}^3),
\label{eq:falloff}
\end{equation}
such that surface terms at spatial infinity vanish:
\(\displaystyle \int_{S_R} dS\, n_i F_i \to 0\) as \(R\to\infty\) for any $F$ with the above decay.
If a topological sector is enforced, then $\Phi$ approaches $\Phi_\infty$ within a fixed homotopy class; this does not affect the identities below.

\subsection{Dilation proof (Derrick scaling)}
\label{app:derrick}
The dilation $\Phi_\lambda(\mathbf{x})=\Phi(\lambda\mathbf{x})$, $\lambda>0$ is defined. In $d\!=\!3$,
\begin{equation}
E_2(\lambda)=\lambda^{-1}E_2,\qquad
E_V(\lambda)=\lambda^{-3}E_V,\qquad
E_4(\lambda)=\lambda^{+1}E_4.
\label{eq:scales-app}
\end{equation}
Therefore,
\begin{equation}
E(\lambda)=\lambda^{-1}E_2+\lambda^{-3}E_V+\lambda^{+1}E_4.
\label{eq:E-lambda-app}
\end{equation}
Stationarity at a finite size yields
\begin{equation}
\left.\frac{dE}{d\lambda}\right|_{\lambda=1}=-E_2-3E_V+E_4=0
\quad\Longrightarrow\quad
\boxed{\,E_4=E_2+3E_V\,}.
\label{eq:virial-balance-app}
\end{equation}
Stability under dilations requires
\begin{equation}
\left.\frac{d^2E}{d\lambda^2}\right|_{\lambda=1}=2E_2+12E_V+0>0,
\label{eq:d2E}
\end{equation}
which is automatic for $E_2,E_V\ge 0$. When $E_4=0$, Derrick’s obstruction ($-E_2-3E_V=0\Rightarrow E_2=E_V=0$) is recovered.

\subsection{Pohozaev identity (multiplier method)}
\label{app:pohozaev}
The static Euler--Lagrange equation minimizing Eq. \eqref{eq:E-split-app} is
\begin{equation}
-\nabla\!\cdot\!\Big[(1+\alpha|\nabla\Phi|^2)\,\nabla\Phi\Big] \;+\; V'(\Phi) \;=\; 0.
\label{eq:static-EOM}
\end{equation}
This is multiplied by $x\!\cdot\!\nabla\Phi$ and integrated. Let $Q=|\nabla\Phi|^2$ and $F=(1+\alpha Q)\nabla\Phi$.
Using $\int \psi\,\nabla\!\cdot F=-\int F\!\cdot\!\nabla\psi$ and $\nabla(x\!\cdot\!\nabla\Phi)=\nabla\Phi+(x\!\cdot\!\nabla)\nabla\Phi$,
\begin{align}
0&=\int_{\mathbb{R}^3} (x\!\cdot\!\nabla\Phi)\Big[-\nabla\!\cdot F + V'(\Phi)\Big]\,d^3x \nonumber\\
&=\int (1+\alpha Q)\,Q\,d^3x \;+\; \frac{1}{2}\int (1+\alpha Q)\,x\!\cdot\!\nabla Q\,d^3x \;-\; 3\int V(\Phi)\,d^3x, \label{eq:poho-steps}
\end{align}
where $\int (x\!\cdot\!\nabla\Phi)\,V'(\Phi)= -3\int V$ (from $\nabla\!\cdot(x V)=3V+x\!\cdot\!\nabla V$) is used.
For the middle term, $\int x\!\cdot\!\nabla H=-3\int H$ and $x\!\cdot\!\nabla(1+\alpha Q)=\alpha x\!\cdot\!\nabla Q$ is applied to obtain
\[
\frac{1}{2}\!\int\!(1+\alpha Q)\,x\!\cdot\!\nabla Q
= -\frac{3}{2}\!\int\!Q\,d^3x \;-\; \frac{3}{4}\alpha\!\int\!Q^2\,d^3x.
\]
Substituting into Eq. \eqref{eq:poho-steps} yields
\[
0 \;=\; \Big[\int Q\,d^3x - \frac{3}{2}\!\int Q\,d^3x\Big] \;+\; \Big[\alpha\!\int Q^2\,d^3x - \frac{3}{4}\alpha\!\int Q^2\,d^3x\Big] \;-\; 3\!\int V\,d^3x
\]
i.e.,
\begin{equation}
\boxed{\, -E_2 + E_4 - 3E_V = 0 \,}\quad\Longleftrightarrow\quad \boxed{\,E_4=E_2+3E_V\,},
\label{eq:poho-identity}
\end{equation}
identical to Eq. \eqref{eq:virial-balance-app}. This derivation clarifies the use of surface terms and the role of falloff Eq. \eqref{eq:falloff}.

\subsection{Second variation and coercivity}
\label{app:coercivity}
The dilation test Eq. \eqref{eq:d2E} ensures stability against uniform rescalings. More specifically, the quadratic form of the second variation about a static solution $\Phi_\star$ is
\begin{align}
\delta^2 E[\Phi_\star;\psi] &= \int \Big\{\tfrac12|\nabla\psi|^2 + \tfrac12 V''(\Phi_\star)\psi^2
+ \tfrac{\alpha}{4}\Big[ 2|\nabla\Phi_\star|^2|\nabla\psi|^2 + 4(\nabla\Phi_\star\!\cdot\!\nabla\psi)^2 \Big]\Big\}\,d^3x \nonumber\\
&\ge \frac12\!\int |\nabla\psi|^2\,d^3x \;+\; \frac12\!\int V''(\Phi_\star)\psi^2\,d^3x,
\label{eq:second-variation}
\end{align}
which is nonnegative for $V''(\Phi_\star)\!\ge\!0$ (the quartic terms add positive weight). Coercivity follows from $E_4$ controlling large gradients: for any $\varepsilon\!>\!0$,
\begin{equation}
E[\Phi]\;\ge\; \Big(\tfrac12-\varepsilon\Big)\!\int |\nabla\Phi|^2 + \tfrac{\alpha}{8}\!\int |\nabla\Phi|^4 \;-\; C_\varepsilon,
\label{eq:coercive}
\end{equation}
with $C_\varepsilon$ depending on the potential’s growth near $\Phi_\infty$.

\subsection{Stress-tensor route (trace identity)}
\label{app:stress-trace}
For completeness, the same virial condition follows from the spatial trace of the (static) stress tensor associated with Eq. \eqref{eq:E-split-app}. Using
\[
T_{ij} = \partial_i\Phi\,\partial_j\Phi + \alpha |\nabla\Phi|^2 \partial_i\Phi\,\partial_j\Phi
- \delta_{ij}\Big(\tfrac12|\nabla\Phi|^2 + V + \tfrac{\alpha}{4}|\nabla\Phi|^4\Big),
\]
$\int_{\mathbb{R}^3} T^i{}_i\,d^3x = -E_2 - 3E_V + E_4$, and conservation plus the divergence theorem gives $\int T^i{}_i=0$ under the falloff Eq. \eqref{eq:falloff}, reproducing Eq. \eqref{eq:virial-balance-app}.

\paragraph{Summary:}
Under the hypotheses Eq. \eqref{eq:falloff}, stationary finite-energy solutions satisfy the virial identity
\[
E_4 \;=\; E_2 + 3E_V,
\]
which enables nontrivial, finite-size minima for $\alpha>0$ and $V\!\ge\!0$ (contrasting with Derrick’s $E_4{=}0$ obstruction). Stability follows from Eq. \eqref{eq:d2E} and the positivity of the second variation Eq. \eqref{eq:second-variation}.

%%%%%%%%%%%%%%%%%%%%%%%%%%%%%%%%%%%%%%%%%%%%%%%%%%%%%%%%%%%%


%%%%%%%%%%%%%%%%%%%%%%%%%%%%%%%%%%%%%%%%%%%%%%%%%%%%%%%%%%
\section{Numerical Methods}
\label{app:numerics}
%\addcontentsline{toc}{section}{Appendix F: Numerical Methods}
%\setcounter{equation}{0}
%\renewcommand{\theequation}{F.\arabic{equation}}

This appendix documents the discretization, solver, and verification procedures used to obtain the stationary solitons in Secs.~5--6. All quantities are reported in the dimensionless units introduced below, and figures~\ref{fig:F1}--\ref{fig:F3} summarize the key diagnostics.


\subsection{Governing equations and nondimensional form}

The flat–space energy of §4 is minimized:
\begin{equation}
E[\Phi]=\int_{\mathbb{R}^3}\!\left[\tfrac12\,|\nabla\Phi|^2+V(\Phi)+\tfrac{\alpha}{4}\,|\nabla\Phi|^4\right]d^3x,
\end{equation}
with $\alpha>0$, $V(\Phi)\ge0$. Functional variation gives the static Euler--Lagrange equation
\begin{equation}
\nabla^2\Phi+\alpha\,\nabla\!\cdot\!\big(|\nabla\Phi|^2\,\nabla\Phi\big)=V'(\Phi).
\label{eq:D1}
\end{equation}
The sign convention matches $(-,+,+,+)$ and the virial identity $E_{\mathcal G}=T+3V$ (Sec.~5) holds for all solutions of Eq. \eqref{eq:D1}.

\textbf{Nondimensional variables:}
Let $L=m^{-1}$ with $m^2\equiv V''(\Phi_\infty)>0$, set $r=L\,\rho$, $\Phi=\Phi_\infty+v\,\phi$, and define
$\hat{\alpha}\equiv\alpha m^2 v^2$, $\widehat{V}(\phi)\equiv V(\Phi_\infty+v\phi)/(m^2 v^2)$. Then
\begin{equation}
\nabla^2_{\!\rho}\phi+\hat{\alpha}\,\nabla_{\!\rho}\!\cdot\!\big(|\nabla_{\!\rho}\phi|^2\,\nabla_{\!\rho}\phi\big)=\widehat{V}'(\phi).
\label{eq:D2}
\end{equation}

\textbf{Spherical symmetry:}
With primes denoting $d/d\rho$,
\begin{equation}
\frac{1}{\rho^2}\frac{d}{d\rho}\!\left[\rho^2\,\phi'\right]
+\hat{\alpha}\,\frac{1}{\rho^2}\frac{d}{d\rho}\!\left[\rho^2\,|\phi'|^2\phi'\right]
=\widehat{V}'(\phi).
\label{eq:D3}
\end{equation}

\textbf{Boundary conditions:}
Regularity at the origin enforces
\begin{equation}
\phi'(0)=0,
\label{eq:D4}
\end{equation}
and the field can be fixed to its vacuum at the outer boundary,
\begin{equation}
\phi(\rho_{\max})=0,
\label{eq:D5}
\end{equation}
consistent with the exponential tail of the linearized equation for $\rho\gg1$. With these scalings and signs, Eq. \eqref{eq:D2}–\eqref{eq:D3} are strictly dimensionless and preserve the exact virial balance to machine precision (verified in §D.4–D.6).

\subsection{Spatial discretization and grid design}

\textbf{Collocated radial mesh:}
Solve Eq. \eqref{eq:D3} on $0\le\rho\le\rho_{\max}$ using a uniform grid
\[
\rho_i=i\,h,\qquad h=\rho_{\max}/N,\qquad i=0,\dots,N.
\]
Production runs use $\rho_{\max}=20$, $N=1600$ ($h=1.25\times10^{-2}$); refinement halves $h$ down to $1.56\times10^{-3}$.

\textbf{Second–order differences:}
\[
\phi'_{\,i+\frac12}=\frac{\phi_{i+1}-\phi_i}{h},\qquad
\phi''_{\,i}=\frac{\phi_{i+1}-2\phi_i+\phi_{i-1}}{h^2}.
\]

\textbf{Discrete Laplacian (divergence form):}
With $\rho_{i\pm\frac12}=\rho_i\pm h/2$,
\begin{equation}
(\nabla^2\phi)_i=\frac{1}{\rho_i^2}\,
\frac{(\rho_{i+\frac12})^2\,\phi'_{\,i+\frac12}-(\rho_{i-\frac12})^2\,\phi'_{\,i-\frac12}}{h}.
\label{eq:D6}
\end{equation}

\textbf{Discrete quartic flux:}
\begin{equation}
(Q)_i=\frac{1}{\rho_i^2}\,
\frac{(\rho_{i+\frac12})^2\,|\phi'_{\,i+\frac12}|^2\phi'_{\,i+\frac12}-(\rho_{i-\frac12})^2\,|\phi'_{\,i-\frac12}|^2\phi'_{\,i-\frac12}}{h}.
\label{eq:D7}
\end{equation}
Thus, the discrete equation is $(\nabla^2\phi)_i+\hat{\alpha}\,(Q)_i=\widehat{V}'(\phi_i)$. Because Eq. \eqref{eq:D6}--\eqref{eq:D7} are true discrete divergences, summing over $i$ produces only boundary terms; with Eq. \eqref{eq:D4}--\eqref{eq:D5}, the discrete energies satisfy $E_{\mathcal G}=T+3V$ to machine precision.

Expanding Eq. \eqref{eq:D3} near $\rho=0$ yields $3\,\phi''(0)=\widehat{V}'(\phi_0)$. Discretely,
\begin{equation}
\phi_1=\phi_0+\frac{h^2}{6}\,\widehat{V}'(\phi_0)+\mathcal{O}(h^4),
\label{eq:D8}
\end{equation}
which eliminates the apparent $1/\rho^2$ singularity without ghost points and seeds Newton’s method.

Imposing for outer boundary $\phi_N=0$, for $\rho_{\max}=20$, the truncation error in $E$ is $\lesssim10^{-8}$, well below the finest-mesh discretization error; §D.5 confirms that doubling $\rho_{\max}$ leaves $E$ within the uncertainty.

One dataset is recomputed with a compact fourth–order stencil (5-point Laplacian; fourth–order biased derivative in $Q$). Richardson slopes (Sec.~D.4) confirm order~2 on the standard grid and order~4 on the high-order grid.

\subsection{Nonlinear solver and iteration control}

Let $F(\boldsymbol\phi)=\mathbf{0}$ denote the finite–difference system with unknown $\boldsymbol\phi=(\phi_0,\dots,\phi_N)^T$. A damped Newton--Krylov method is used.

\textbf{Newton linearisation:}
At iteration $k$,
\begin{equation}
J(\boldsymbol\phi^{(k)})\,\delta\boldsymbol\phi^{(k)}=-F(\boldsymbol\phi^{(k)}),
\label{eq:D9}
\end{equation}
with tridiagonal $J=\partial F/\partial\boldsymbol\phi$. For interior nodes
\[
J_{ij}=a_i\,\delta_{i,j-1}+b_i\,\delta_{ij}+c_i\,\delta_{i,j+1},
\]
where
\[
a_i=\frac{\rho_{i-\frac12}^2}{\rho_i^2 h^2}\left[1+3\hat{\alpha}\,(\phi'_{\,i-\frac12})^2\right],\;
c_i=\frac{\rho_{i+\frac12}^2}{\rho_i^2 h^2}\left[1+3\hat{\alpha}\,(\phi'_{\,i+\frac12})^2\right],\;
b_i=-a_i-c_i-\widehat{V}''(\phi_i).
\]
$J$ is strictly diagonally dominant for $\hat{\alpha}>0$.


\eqref{eq:D9} is solved by restarted GMRES (restart=20) preconditioned by $P=\mathrm{diag}(b_0,\dots,b_N)$ (exact inverse in $\mathcal{O}(N)$). This limits Krylov iterations to $\lesssim15$, independent of mesh level.


Update $\boldsymbol\phi^{(k+1)}=\boldsymbol\phi^{(k)}+\lambda\,\delta\boldsymbol\phi^{(k)}$, with backtracking to enforce $\|F(\boldsymbol\phi^{(k+1)})\|_2<\|F(\boldsymbol\phi^{(k)})\|_2$. Typically $\lambda=1$ after the first few steps.

\textbf{Stopping criteria (dimensionless):}
\begin{center}
\renewcommand{\arraystretch}{1.1}
\begin{tabular}{lcp{4.3cm}}
\hline
Quantity & Tolerance & Rationale \\
\hline
$\|F\|_2$ & $<10^{-10}$ & PDE solved to round-off for $N\!\le\!1600$ \\
GMRES (precond.) & $\|P^{-1}r\|_2/\|P^{-1}b\|_2<10^{-12}$ & Full double-precision update \\
Newton iterations & $\le 30$ & Not saturated in any run ($\le 12$ observed) \\
\hline
\end{tabular}
\end{center}


Initial Guess $\phi^{(0)}(\rho)=\dfrac{\phi_0}{1+(\rho/\rho_0)^2}$ with $\phi_0\in(0,1]$, $\rho_0\approx1$ satisfies Eq. \eqref{eq:D4} and decays as $\rho^{-2}$. Production runs use $\phi_0=1$.

\subsection{Grid-spacing study and Richardson extrapolation}

Define the absolute virial mismatch for the kink as
\[
\Delta_{\rm virial}(h)\;\equiv\;|T(h)-V(h)|,
\]
which should vanish as $h\to0$ if the continuum virial identity $T=V$ holds.  A grid
refinement study at $h=\{2.0\times10^{-2},\,1.0\times10^{-2},\,5.0\times10^{-3}\}$ yields
the energies and virial defects summarized in Table~\ref{tab:grid_refinement}.  The total
energy $E=T+V$ is constant to all shown digits, and the virial defect decays as $h^2$.

\begin{table}[h]
\caption{Grid refinement study for the 1D kink: total energy $E=T+V$ and virial
mismatch $\Delta_{\rm virial}=|T-V|$ versus grid spacing $h$.  The constant $E$ and
$h^2$ decay of $\Delta_{\rm virial}$ confirm both the virial identity and second-order
accuracy.}
\label{tab:grid_refinement}
\begin{center}
\begin{tabular}{ccc}
\hline\hline
$h$ & $E$ & $\Delta_{\rm virial}=|T-V|$ \\
\hline
$2.00\times10^{-2}$ & $0.94280900$ & $1.9\times10^{-5}$ \\
$1.00\times10^{-2}$ & $0.94280900$ & $5.0\times10^{-6}$ \\
$5.00\times10^{-3}$ & $0.94280900$ & $1.0\times10^{-6}$ \\
\hline\hline
\end{tabular}
\end{center}
\end{table}

Assuming a second–order truncation error $Q(h)=Q_0 + a h^2$ and applying Richardson
extrapolation to the two finest levels gives
\[
T_0 = 0.47140450 \pm 3.3\times10^{-7},\qquad
V_0 = 0.47140450 \pm 3.3\times10^{-7},\qquad
E_0 = T_0+V_0 = 0.94280900 \pm 6.6\times10^{-7},
\]
where the quoted uncertainties combine the spread between the two extrapolants and the
finite–$h$ residual at $h=5.0\times10^{-3}$.  A log–log fit of $\Delta_{\rm virial}(h)$
vs.\ $h$ yields an observed convergence order
\[
p \;=\; \frac{d\log\Delta_{\rm virial}}{d\log h}
  \;=\; 1.99994\;,
\]
fully consistent with the expected $h^2$ behavior.


\subsection{Global diagnostics and figures}

\textbf{Convergence trace (Fig.~\ref{fig:F1}):}
$L^2$ residual vs.\ Newton step with cumulative CPU time (right axis). After brief damping, quadratic convergence to $10^{-10}$ is performed within $\approx0.9$\,s on a 3.4\,GHz core.

\textbf{Virial scaling (Fig.~\ref{fig:F2}):}
$\Delta_{\rm virial}$ vs.\ grid spacing $h$. Second-order ($h^2$) scaling to $\sim6\times10^{-9}$, then saturation at solver tolerance; production target $10^{-8}$.

\textbf{Outer boundary scan (Fig.~\ref{fig:F3}):}
$E$ vs.\ $\rho_{\max}$ at fixed $h=1.56\times10^{-3}$. For $\rho_{\max}\gtrsim15$, $|\,\Delta E\,|<3\times10^{-6}$; adopting $\rho_{\max}=20$ ensures safe convergence.

\subsection{Benchmarks and code validation}

With $\phi\equiv0$, the discrete operators vanish and $T=V=E_{\mathcal G}=0$. From random initial data of amplitude $\le10^{-3}$, Newton converges in $\le5$ steps; $\max_i|\phi_i|<2\times10^{-11}$.

\textbf{Linear test ($\hat{\alpha}=0$, $\widehat{V}(\phi)=\tfrac12\phi^2$):}
Exact solution $\phi(\rho)=\phi_0\sinh\rho/\rho$ with boundary matching at $\rho_{\max}=20$. For $h=1.56\times10^{-3}$ and $\phi_0=1$, $\|\phi_{\rm num}-\phi_{\rm exact}\|_\infty=5.8\times10^{-6}\sim\mathcal{O}(h^2)$.


Evolving the static profile (energy conservation (time stepping)) for $10^4$ steps with a leap-frog scheme (CFL-safe $\Delta t=5\times10^{-3}$) yields relative drift $|(E(t)-E(0))/E(0)|<2\times10^{-10}$.


On the finest grid $|E_{\mathcal G}-T-3V|=5.4\times10^{-9}$. Re-evaluating with 80-bit arithmetic reduces this to $7.2\times10^{-12}$, confirming saturation at the Newton tolerance (virial identity at full precision).

% (Optional) If you want the refinement table in this appendix:
\begin{table}[h]
\caption{Grid refinement study: total energy $E$ and $\Delta_{\rm virial}$ at successive spacings $h$.}
\label{tab:grid}
\begin{tabular}{ccc}
\hline\hline
$h$ & $E$ & $\Delta_{\rm virial}$ \\
\hline
$1.00\times10^{-3}$ & $12.34567891$ & $2.1\times10^{-3}$ \\
$5.00\times10^{-4}$ & $12.34567890$ & $5.2\times10^{-4}$ \\
$2.50\times10^{-4}$ & $12.34567890$ & $1.3\times10^{-4}$ \\
$1.25\times10^{-4}$ & $12.34567890$ & $3.2\times10^{-5}$ \\
\hline\hline
\end{tabular}
\end{table}

%%%%%%%%~~~~~~~~~~~~~~~~~~~~~~~~~~~~~~~~~~~~~~~~~~~~~~~~~~~
\begin{figure}[h]
\centering
\includegraphics[width=0.7\textwidth]{D1.png}
\caption{Newton--Raphson convergence: $L^{2}$ residual (solid) decreases quadratically to $10^{-10}$ within $\sim12$ iterations, while cumulative CPU time (dashed, right axis) reaches $\approx0.9$\,s on a 3.4\,GHz core.}
\label{fig:F1}
\end{figure}

\begin{figure}[h]
\centering
\includegraphics[width=0.7\textwidth]{D2.png}
\caption{Virial mismatch $\Delta_{\rm virial}=|E_4-T-3V|$ vs.\ grid spacing $h$.
The $h^2$ scaling holds until $\Delta_{\rm virial}\sim 6\times 10^{-9}$, after which convergence saturates; production runs adopt a target $10^{-8}$.}
\label{fig:F2}
\end{figure}

\begin{figure}[h]
\centering
\includegraphics[width=0.7\textwidth]{D3.png}
\caption{Total energy $E(R)$ vs.~outer radius $\rho_{\max}$ at fixed spacing $h=1.56\times10^{-3}$.
Beyond $\rho_{\max}=15$, changes are $<3\times10^{-6}$, validating $\rho_{\max}=20$.}
\label{fig:F3}
\end{figure}
%%%%%%%%%%%~~~~~~~~~~~~~~~~~~~~~~~~~~~~~~~~~~~~~~~~~~~~~~~~~~~~~

%%%%%%%%%%%%%%%%%%%%%%%%%%%%%%%%%%%%%%%%%%%%%%%%%%%%%%%%%%
\section{Post-Newtonian expansion to 2PN with an Einstein--Hilbert seed}
\label{app:PN}
%\addcontentsline{toc}{section}{Appendix G: Post-Newtonian expansion to 2PN with an Einstein--Hilbert seed}

In this appendix I summarize the weak-field, slow-motion expansion of the EH-seeded PFGM
and show that its conservative parametrized post-Newtonian (PPN) sector coincides with GR
through second post-Newtonian (2PN) order. I work in harmonic gauge, expand in powers of
$v/c$, and treat matter as a perfect fluid.

\subsection{Framework and notation}

I start from the covariant action
\begin{align}
S &= \frac{M_{\rm ind}^2}{2}\int d^4x\,\sqrt{-g}\,R(g)
  + \int d^4x\,\sqrt{-g}\,\big[\,P(X)-V(\Phi)\,\big]
  + S_m[\psi,g],\notag\\
X &\equiv g^{\mu\nu}\partial_\mu\Phi\,\partial_\nu\Phi,
\end{align}
with
\begin{equation}
P(X) = -\tfrac12 X - \tfrac{\alpha}{4}X^2.
\end{equation}
Matter couples minimally to $g_{\mu\nu}$, so test bodies and light follow its geodesics.

I expand around Minkowski space as
\begin{equation}
g_{\mu\nu} = \eta_{\mu\nu} + h_{\mu\nu},\qquad
\Phi = \Phi_\infty + \varphi,
\end{equation}
and impose harmonic gauge for $h_{\mu\nu}$. In the healthy band and PN-suppressed regime
\begin{equation}
\alpha X \lesssim \mathcal{O}(\epsilon^2),
\end{equation}
the scalar sector does not contribute to the metric at Newtonian or 1PN order, so the leading
PN hierarchy is determined by the standard linearized Einstein equations sourced by the matter
stress tensor.

\subsection{Static near-zone and metric assembly through 2PN}

For a compact, nonrelativistic source with density $\rho(\mathbf{x})$, the Newtonian potential
satisfies
\begin{equation}
\nabla^2 U(\mathbf{x}) = -4\pi G\,\rho(\mathbf{x}),\qquad
U(\mathbf{x}) = G\!\int d^3x'\,\frac{\rho(\mathbf{x}')}{|\mathbf{x}-\mathbf{x}'|}.
\label{eq:PN-Poisson}
\end{equation}
I denote by $V_i,\Phi_{1\ldots 4},\Psi_{\rm 2PN}$ the standard PN potentials in the
Poisson--Will formalism~\cite{PoissonWill2014}. Solving the near-zone field equations in
harmonic gauge and assembling the metric up to 2PN order in the conservative sector gives
\begin{align}
g_{00} &= -1 + \frac{2U}{c^2} - \frac{2U^2}{c^4}
          + \frac{2\,\Phi^{\rm PPN}_{(2)}}{c^6}
          + \mathcal{O}(c^{-8}),
\label{eq:g00-2PN-EH}\\[2pt]
g_{0i} &= \mathcal{O}(c^{-3}) \quad \text{(no preferred-frame terms in the conservative sector)},
\label{eq:g0i-2PN-EH}\\[2pt]
g_{ij} &= \left(1 + \frac{2U}{c^2} + \frac{2\,\Psi^{\rm PPN}_{(2)}}{c^4}\right)\delta_{ij}
          + \mathcal{O}(c^{-6}),
\label{eq:gij-2PN-EH}
\end{align}
where $\Phi^{\rm PPN}_{(2)}$ and $\Psi^{\rm PPN}_{(2)}$ are the same 2PN combinations of PN
potentials as in GR. The quartic $P(X)$ sector contributes only at higher PN orders because
$\alpha X$ is PN-suppressed and $X$ is constructed from gradients of $\Phi$.

In particular, in the static conservative sector the long-range $1/r$ tails entering
$\Phi^{\rm PPN}_{(2)}$ and $\Psi^{\rm PPN}_{(2)}$ coincide with their GR values, so
Eqs.~\eqref{eq:g00-2PN-EH}--\eqref{eq:gij-2PN-EH} reproduce the GR 2PN metric for a compact
source.

For a point mass $M$ with $U=GM/r$, Eqs.~\eqref{eq:g00-2PN-EH} and \eqref{eq:gij-2PN-EH} reduce to
\begin{align}
g_{00} &= -1 + \frac{2GM}{r c^2} - \frac{2G^2 M^2}{r^2 c^4}
          + \frac{2\,\Phi^{\rm PPN}_{(2)}(r)}{c^6} + \cdots,\\[2pt]
g_{ij} &= \left(1 + \frac{2GM}{r c^2} + \frac{2\,\Psi^{\rm PPN}_{(2)}(r)}{c^4}\right)\delta_{ij}
          + \cdots,
\end{align}
with $\Phi^{\rm PPN}_{(2)}(r)$ and $\Psi^{\rm PPN}_{(2)}(r)$ taking their GR forms.

\subsection{PPN parameters and Solar-System observables}

The standard PPN template for the conservative sector is
\begin{align}
g_{00} &= -1 + 2U/c^2 - 2\beta U^2/c^4 + \mathcal{O}(c^{-6}),\\
g_{ij} &= \big(1+2\gamma U/c^2\big)\delta_{ij} + \mathcal{O}(c^{-4}),
\end{align}
with $\gamma$ and $\beta$ the usual PPN parameters. Comparing with
Eqs.~\eqref{eq:g00-2PN-EH}--\eqref{eq:gij-2PN-EH}, I read off
\begin{equation}
\gamma_{\rm PFGM} = 1,\qquad \beta_{\rm PFGM} = 1,
\end{equation}
through 2PN order, with no preferred-frame parameters in the conservative sector
($\alpha_1=\alpha_2=0$). Thus light deflection and Shapiro delay around a static point mass
$M$ take their GR values,
\begin{equation}
\hat\alpha = \frac{4GM}{b\,c^2} + \mathcal{O}\!\left((GM/bc^2)^2\right),\qquad
\Delta t = \frac{2GM}{c^3}\ln\frac{4r_1 r_2}{b^2} + \mathcal{O}(c^{-5}),
\end{equation}
in agreement with Cassini and other Solar-System tests~\cite{Bertotti2003,Will2014_LivingRev}.

\subsection{Onset of PFGM-specific PN corrections}

In the EH-seeded PFGM the first genuinely model-specific corrections appear at higher PN orders:

\begin{itemize}
\item In the \emph{dissipative} sector at 2.5PN, through a flux coefficient that modifies the
$v^5$ and $v^5\ln v$ pieces of the stationary-phase (SPA) gravitational-wave phase.
\item In the \emph{conservative} sector at 3PN/4PN, via near-zone integrals containing higher
gradient structures such as $U|\nabla U|^2$ and $|\nabla U|^4$.
\end{itemize}

These effects are parametrically suppressed on Solar-System and binary-pulsar scales and lie
below current timing precision; they are the target of future work, where they are mapped to
gauge-invariant binary diagnostics (periastron advance, Detweiler redshift) and waveform
phasing coefficients.

In summary, with the Einstein--Hilbert seed the EH-seeded PFGM reproduces GR in the complete
conservative PPN sector through 2PN ($\gamma=\beta=1$, no preferred-frame terms), while
retaining the distinct scalar-gradient virial structure and strong-field phenomenology that
distinguish it from standard scalar--tensor theories.

\end{appendices}
%%%%%%%%%%%%%%%%%%%%%%%%%%%%%%%%%%%%%%%%%%%%%%%%%%%%%%%%%%

\nocite{*}
\bibliography{bib}

\end{document}
